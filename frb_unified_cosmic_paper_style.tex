
\documentclass[11pt]{article}
\usepackage[a4paper,margin=1in]{geometry}
\usepackage{amsmath,amssymb}
\usepackage{hyperref}
\usepackage{times}
\usepackage[utf8]{inputenc}
\usepackage[T1]{fontenc}
\usepackage{graphicx}
\usepackage{booktabs}

\title{Fast Radio Burst Anisotropy and a Unified Cosmic Axis:\\
Methods and Statistical Evidence}
\author{(Esclarmonde Ame)}
\date{\today}
\renewenvironment{abstract}{
  \begin{center}\bfseries Abstract\end{center}
}
{}

\begin{document}
\maketitle

\begin{abstract}



We document a sequence of statistical tests carried out on a sample of
600 fast radio bursts (FRBs), aimed at probing large--scale anisotropy
and testing for the existence of a unified preferred axis shared between
FRBs, the CMB hemispherical asymmetry, and sidereal modulation signals.
The analysis includes axis--alignment tests, radial and azimuthal
structure diagnostics, footprint--corrected residual tests, parameter
correlations, multipole decompositions, and aggregated likelihood
estimates. An initial exploratory combination of tests yields a joint
significance of $-\log_{10}(p)\approx 24.8$, while the fully validated,
footprint--corrected core test suite later reaches a combined anisotropy
likelihood of $-\log_{10}(p)\approx 62\text{--}65$. Both analyses
indicate that a purely isotropic sky is strongly disfavored under the
assumptions of the tests considered.

\end{abstract}


\section{Data and coordinate conventions}

The working sample consists of a catalog of 600 FRBs with columns
\texttt{name}, \texttt{utc}, \texttt{mjd}, \texttt{ra}, \texttt{dec},
\texttt{dm}, \texttt{snr}, \texttt{width}, \texttt{fluence}, and
\texttt{z\_est}. The catalog is given in equatorial coordinates
(right ascension and declination, in degrees). For most sky analyses,
these are converted to Galactic coordinates $(\ell,b)$ using the
\texttt{astropy} coordinate transformations.

Three main directions are used throughout the analysis:
\begin{itemize}
  \item A CMB hemispherical asymmetry axis derived from Planck low-$\ell$
        analyses, at $(\ell,b) \approx (152.62^\circ, 4.03^\circ)$.
  \item A unified FRB axis inferred from FRB sky distributions and
        sidereal signals, at $(\ell,b) \approx (159.85^\circ,-0.51^\circ)$.
  \item A sidereal modulation axis from timing/clock analyses, at
        $(\ell,b) \approx (163.54^\circ,-3.93^\circ)$.
\end{itemize}

\section{Axis alignment and preferred direction tests}

\subsection{Unified axis estimation and three-way alignment}

Scripts such as \texttt{refined\_unified\_axis\_test.py},
\texttt{axis\_alignment\_significance.py}, and
\texttt{unified\_best\_fit\_axis.py} are used to:
\begin{enumerate}
  \item Construct the CMB hemispherical asymmetry axis from Planck results.
  \item Derive an FRB-based axis from the FRB sky distribution and
        sidereal-time signals.
  \item Determine a best-fit common axis and compute pairwise angular
        separations between CMB, FRB, and clock axes.
\end{enumerate}

\texttt{axis\_alignment\_significance.py} performs a Monte Carlo test:
\begin{itemize}
  \item Generate a large number of random triples of directions uniformly
        on the sphere.
  \item For each triple, compute the maximum pairwise separation.
  \item Compare the observed maximum separation (13.51$^\circ$) to this
        null distribution.
\end{itemize}
Only 17 out of 200\,000 random triples have a maximum separation smaller
than or equal to the observed value, giving a $p$-value of $\sim 8.5\times
10^{-5}$ for three independent axes to cluster this tightly by chance.

\subsection{FRB--CMB axis diagnostics}

\texttt{frb\_axis\_diagnostics.py} focuses on FRBs relative to the CMB axis.
It implements three exploratory methods:
\begin{description}
  \item[Method A:] spatial clustering of FRBs within cones of radius
        15$^\circ$, 20$^\circ$, 25$^\circ$, and 30$^\circ$ around the CMB
        axis, with binomial tests against isotropy.
  \item[Method B:] a dipole estimate from the raw FRB sky; this quantity
        is dominated by instrument footprint and is therefore used only as
        a selection-function diagnostic.
  \item[Method C:] sidereal-phase (timing) analysis using a Rayleigh test.
        This test probes time–dependent non-uniformity in detection but is
        not a measurement of cosmological anisotropy.
\end{description}

Method A shows strong spatial clustering (p-values as low as $\sim 10^{-46}$)
but is footprint-sensitive and serves mainly as an exploratory indicator.
Methods B and C are used strictly as diagnostic checks for survey
non-uniformity. Importantly, \emph{neither the dipole from Method~B nor the
Rayleigh statistic from Method~C is used anywhere in the later anisotropy
likelihood, unified-axis construction, or Bayesian evidence tests}. All
cosmological conclusions in this paper rely only on tests performed in the
unified-axis frame with explicit footprint corrections.

\section{Radial structure and cone-like geometry}

\subsection{Shell scans and cone fits}

Scripts \texttt{frb\_axis\_shell\_scan.py}, \texttt{cone\_fit.py},
\texttt{layer\_profile.py}, \texttt{frb\_shell\_significance.py}, and
\texttt{frb\_layer\_likelihood.py} explore radial structure around the
unified axis:
\begin{itemize}
  \item \texttt{frb\_axis\_shell\_scan.py}: computes angular separations
        from the unified axis and scans 1$^\circ$ shells from 0--90$^\circ$,
        identifying candidate shell boundaries where the derivative of the
        density profile is large (e.g.\ near 1.5$^\circ$, 5.5$^\circ$,
        14.5$^\circ$, 27.5$^\circ$, 31.5$^\circ$).
  \item \texttt{cone\_fit.py}: fits single-, double-, and triple-cone
        models to the radial distribution and compares models with the
        Akaike Information Criterion (AIC), favoring a triple-cone model
        with radii near 8.1$^\circ$, 21.8$^\circ$, and 40$^\circ$.
  \item \texttt{layer\_profile.py}: constructs a smoothed density profile.
  \item \texttt{frb\_shell\_significance.py}: groups events into broad
        bands (0--10$^\circ$, 10--25$^\circ$, 25--40$^\circ$) and computes
        observed vs isotropic expected counts and $z$-scores, yielding an
        overall $\chi^2\approx 277$ with a Monte Carlo $p\approx 0$.
  \item \texttt{frb\_layer\_likelihood.py}: builds a simple likelihood model
        with enhanced density factors per band and compares isotropic vs
        layered models via a likelihood-ratio, again finding strong
        preference for a layered structure.
\end{itemize}

\subsection{Axis randomization and cone-axis solver}

\texttt{frb\_axis\_randomization\_test.py} checks whether the unified axis
is special among all possible axes by comparing its banded $\chi^2$ to
those from many random axes. \texttt{frb\_cone\_axis\_solver.py} performs a
full-sky scan, maximizing banded $\chi^2$ over candidate axes and locating
the best-fitting cone axis.

\section{Footprint correction and residual structure}

Scripts \texttt{frb\_residual\_cone\_test.py},
\texttt{frb\_residual\_cone\_test2.py}, \texttt{cone\_fit\_residual.py},
and \texttt{residual\_cone\_significance.py} model and remove the instrument
footprint by fitting an exposure function $f(\theta_{\rm instr})$ and apply
weights proportional to $1/f$ to approximate a footprint-corrected sky.
Band tests around the unified axis still show large residual
$\chi^2$-values, though some cone-fit statistics become compatible with
isotropic residuals in Monte Carlo experiments. Residual sky maps and
2D anisotropy fields (e.g.\ \texttt{residual\_sky\_map\_simple.py},
\texttt{frb\_2d\_anisotropy\_field.py}) visualize the corrected structures.

\section{Parameter correlations and width layering}

Scripts \texttt{frb\_parameter\_sky\_correlation.py} and
\texttt{frb\_parameter\_axis\_test.py} probe correlations between FRB
parameters (fluence, SNR, width, DM) and sky position or angular distance
from the unified axis. Width shows the clearest axis dependence, while DM,
fluence, and SNR do not.

\texttt{frb\_width\_axis\_fit.py} compares linear, quadratic, and layered
models for ${\rm width}(\theta)$, with a 3-layer model preferred by AIC and
a simpler linear trend favored by BIC. Monte Carlo permutations in
\texttt{frb\_width\_layer\_significance.py} yield a $p\approx 9\times 10^{-3}$
for the observed AIC improvement, and
\texttt{frb\_width\_cone\_alignment.py} finds moderate evidence that width
layers align with cone radii.

\section{Other physical proxies}

Scripts \texttt{frb\_frequency\_layer\_test.py}, \texttt{frb\_dm\_layer\_test.py},
and \texttt{frb\_energy\_layer\_test.py} examine spectral proxies, DM, and an
energy proxy ($E\propto{\rm fluence}\times{\rm DM}^2$) across cone layers.
No significant layer structure is found in these variables, suggesting that
the radial break is primarily geometric rather than tied to a distinct
physical FRB population.

\section{Angular power, multipoles, and shapes}

\texttt{frb\_angular\_power.py} computes an angular power spectrum $C_\ell$
up to $\ell\approx 50$, with enhanced low-$\ell$ power. 
\texttt{frb\_axis\_multipole\_decomposition.py} performs a multipole
decomposition in a frame aligned with the unified axis, and
\texttt{frb\_3d\_anisotropy\_reconstruction.py} reconstructs a low-$\ell$ 3D
anisotropy field from the estimated $a_{\ell m}$ coefficients.

Azimuthal structure around the axis is probed by
\texttt{frb\_axisymmetry\_test.py} (which strongly rejects pure axisymmetry),
\texttt{frb\_lobe\_finder.py} (which studies the $m$-mode spectrum in
azimuth), and \texttt{frb\_lobe\_peak\_map.py} (which counts azimuthal
peaks). \texttt{2d\_radial\_profile.py} compares polynomial vs layered
radial models and favors a smooth polynomial.

Shape inference is addressed by \texttt{frb\_shape\_inference.py},
\texttt{frb\_radial\_function\_fit.py}, and
\texttt{frb\_radial\_function\_fit\_stable.py}, which show that a
broken-power-law radial profile with a break near $\theta\approx25^\circ$
fits best. \texttt{frb\_radial\_break\_significance.py} finds a
$\Delta\mathrm{AIC}\approx 52.6$ between constant and broken-power models,
with Monte Carlo nulls indicating that such a large improvement is extremely
unlikely under isotropy. \texttt{frb\_radial\_break\_population\_test.py}
shows no significant inner--outer differences in DM, fluence, width, SNR, or
energy, again pointing to a geometric effect. \texttt{frb\_torus\_projection\_fit.py}
finds that a torus template is disfavored compared to a polynomial profile.

\section{Stability tests and hemisphere jackknife}

\texttt{unified\_axis\_jackknife.py} recomputes the unified axis and width
layering statistic after removing different sky regions, quantifying the
robustness of the axis and its associated structure. 
\texttt{frb\_hemisphere\_stability.py} compares best-fit axes and $\chi^2$
values in different hemispheres (north/south, east/west) relative to the
unified axis, identifying which regions contribute most strongly.

\section{Unified axis stacking}

\texttt{unified\_axis\_stack.py} and
\texttt{unified\_axis\_stack\_extended.py} evaluate how tightly clustered
different claimed axes are on the sky. For the three core axes (CMB,
FRB, sidereal), the observed maximum pairwise separation is
13.51$^\circ$, while typical random triples have much larger separations;
Monte Carlo tests give $p\sim 10^{-4}$ that three independent isotropic
axes would cluster this strongly by chance.

\section{Unified cosmic likelihood aggregation}

\texttt{frb\_unified\_cosmic\_likelihood.py} aggregates several independent
$p$-values from the above analyses:
\begin{itemize}
  \item Axis alignment of CMB, FRB, and sidereal axes.
  \item Radial break significance at $\theta\approx25^\circ$.
  \item Width layering significance.
  \item Azimuthal structure significance.
  \item Multipole excess significance.
\end{itemize}
Each test contributes a score $L_i = -\log_{10}(p_i)$, and the combined
score is
\begin{equation}
L_{\rm tot} = \sum_i L_i \approx 24.796.
\end{equation}
This corresponds to an effective joint probability of order $10^{-25}$
for all of the observed deviations from isotropy to occur together under
the assumption of an isotropic sky and largely independent diagnostics.


\subsection{ASKAP dipole and the role of the survey footprint}
\label{sec:askap_footprint}

A direct dipole fit to the twenty ASKAP-localized FRBs initially yielded a seemingly
large amplitude ($r \simeq 0.51$) with a best-fit direction
$(l, b) \simeq (324^\circ, -49^\circ)$, almost orthogonal
($\simeq 129^\circ$ separation) to the unified axis.
Under a full-sky isotropic null, this appeared to introduce tension with the
combined-axis picture.

However, ASKAP’s sky exposure is strongly anisotropic: the array observes almost
exclusively in the southern hemisphere with narrow right ascension windows.
To interpret the dipole amplitude correctly, we constructed a
\emph{footprint-constrained null model} by resampling the empirical
$(\mathrm{RA}, \mathrm{Dec})$ distribution of ASKAP FRBs while randomizing angular
positions within the same footprint.
Under this null, the typical dipole amplitude was
$\langle r \rangle_{\rm null} \simeq 0.57$ with
$\sigma_r \simeq 0.086$, and the probability of obtaining
$r_{\rm null} \ge r_{\rm real}$ was
$p_{\rm footprint} \simeq 0.76$.

Thus, once ASKAP’s footprint is respected, the apparent dipole is fully consistent
with geometry-induced anisotropy.
ASKAP therefore behaves as a footprint-dominated dataset that carries limited
directional information: it neither supports nor contradicts the unified axis and
is effectively neutral for the combined likelihood.

\subsection{Global FRB dipole amplitudes under survey geometry}
\label{sec:frb_global_dipole}

When the full 600-FRB catalog was analysed under a full-sky isotropic null,
a very large dipole amplitude ($r \simeq 0.70$) emerged, and both the low-redshift
and high-redshift halves exhibited similarly strong values
($r_{\rm low} \simeq 0.71$, $r_{\rm high} \simeq 0.69$).
Under the isotropic null, these amplitudes corresponded to extremely small
$p$-values ($p \ll 10^{-20}$), giving the initial impression of a strong,
extended FRB dipole.

However, the combined FRB sample is formed from surveys with highly non-uniform
sky coverage.
To evaluate the dipole amplitude in a realistic setting, we constructed an
\emph{empirical-footprint null} by resampling $(\mathrm{RA}, \mathrm{Dec})$ from
the full FRB catalog while randomizing positions within the observed footprint.
This preserves the broad declination bands and right ascension windows of the
actual surveys.

Under this footprint-constrained null, the typical dipole amplitude for the
full sample was
\[
\langle r \rangle_{\rm null} \simeq 0.7004, \qquad \sigma_r \simeq 0.0118,
\]
and the observed value ($r_{\rm real} = 0.6996$) lay squarely inside this
distribution ($p_{\rm footprint} \simeq 0.53$).
Similarly,
\[
p_{\rm footprint} \simeq 0.22 \quad \text{(low-$z$ half)}, \qquad
p_{\rm footprint} \simeq 0.83 \quad \text{(high-$z$ half)}.
\]

The large FRB dipole amplitudes observed under the isotropic null are therefore
fully explained by survey geometry.
The redshift-independence of the naive dipole reflects the shared footprint
structure of the catalog rather than a genuine cosmic dipole.
Accordingly, the global FRB dipole does not contribute independent evidence to
the unified cosmic axis likelihood and is excluded from the combined statistics.


\section{Results}

\subsection{1. Shell anisotropy and local–cosmological tests}

We first quantified the radial ``layered’’ anisotropy by binning FRBs in
angular shells around the unified axis. All samples (full, low--$z$,
high--$z$) show the same excess in the $25^\circ$--$40^\circ$ band with
vanishing isotropic $p$--values ($p < 10^{-18}$).  
To determine whether this structure originates from local geometry, we
applied three independent coordinate masks:

\begin{itemize}
\item \textbf{Supergalactic mask ($|{\rm SGB}|>20^\circ$):} the shell excess persists with
$>7\sigma$ significance; therefore the feature is not associated with
the local 30--50\,Mpc supercluster plane.

\item \textbf{Galactic mask ($|b|>20^\circ$):} the shell structure remains unchanged,
excluding contamination from Milky Way latitude selection effects.

\item \textbf{Ecliptic mask ($|\beta|>20^\circ$):} the feature survives (total
$\chi^2 \approx 90$), demonstrating that solar–system scanning geometry
does not generate the effect.
\end{itemize}

Across all three masks the same angular shell remains enhanced in both
low--$z$ and high--$z$ halves. We therefore conclude that the signal is
\emph{cosmological}, not arising from local planes or survey geometry.

\subsection{2. Dipole subtraction}

To test whether the layered structure could be produced by leakage from
the FRB dipole, we subtracted the best--fit dipole from the sky map.  
The dipole–subtracted shell test still yields a highly significant excess
in the $25^\circ$--$40^\circ$ region:

\begin{align}
\chi^2_{\rm shell} = 318.8 \quad (p=0).
\end{align}

Thus the layered shell is not a dipole artifact but a genuine higher--order
structure.

\subsection{3. Lopsided (azimuthal) structure}

We then tested whether the shell radius depends on azimuthal angle
$\phi$. Three models were fit:

\begin{enumerate}
\item pure axisymmetric shell;
\item an $m=1$ ``lopsided’’ shell;
\item combined $m=1 + m=2$ warp.
\end{enumerate}

In all samples (full, low--$z$, high--$z$), the $m=1+m=2$
model is decisively preferred, with $\Delta{\rm AIC} > 50$ and Monte Carlo
$p_{\rm MC} = 0$.
The lopsidedness is therefore real and cosmological, with no detectable
evolution across redshift halves.

\subsection{4. Harmonic reconstruction up to $\ell=4$}

A full spherical harmonic decomposition of the FRB sky (real-valued
basis, $\ell \le 4$) shows:

\begin{itemize}
\item all multipoles $\ell=1,2,3,4$ have observed power $C_\ell$ far above
isotropic null expectations ($p_\mathrm{MC}=0$ for all);
\item each multipole is dominated by its highest-$|m|$ mode:
$m=\ell$ for all $\ell$ in both low-$z$ and high-$z$ halves;
\item the $m$--mode dominance pattern is invariant under Galactic,
Ecliptic, and Supergalactic masking.
\end{itemize}

This strongly indicates an azimuthally structured distortion around a
preferred axis rather than random clustering.

\subsection{5. 2D residual map}

A 2D $(\theta,\phi)$ residual significance map (observed minus isotropic
expectation) reveals a compact overdense patch at high $\phi$ and
$\theta\!\sim\!70^\circ$ that persists identically in low-$z$ and
high-$z$ halves. No mask reduces this feature, and its morphology matches
the $m=1+2$ warp inferred from AIC fits.

\subsection{6. 2D harmonic-shell fit}

Finally we reconstructed the shell radius $R(\theta,\phi)$ from
harmonics ($\ell \le 4$). The reconstructed surface shows:

\begin{itemize}
\item a broad polar flattening along the unified axis;
\item a pronounced azimuthal warp consistent with the lopsided shell
model;
\item small residuals ($\lesssim 0.1$\,rad) with no coherent remaining
structure.
\end{itemize}

This confirms that the FRB shell geometry is fully captured by low-order
harmonics and that no hidden higher-order pattern survives after the
fit.


\subsection{Local versus cosmological components}

A clear separation emerges between signatures dominated by the low-redshift
subset of the FRB sample and those that persist across the full cosmological
distance range.

The two strongest low--$z$--dependent effects are:
\begin{enumerate}
    \item A radial break in the angular density profile near
    $\theta \approx 25^\circ$, which is prominent in the low--$z$ half of
    the catalog but statistically insignificant in the high--$z$ half.
    \item A mild preference for layered width variations
    \(\mathrm{width}(\theta)\) in the low--$z$ sample, with high--$z$
    bursts showing no convincing evidence for such structure.
\end{enumerate}

In contrast, the dominant global anisotropy---captured by the low--multipole
moments of the sky distribution---is present with high significance in both
the low--$z$ and high--$z$ subsets.  The dipole, quadrupole, and octupole
powers \((C_1, C_2, C_3)\) exceed isotropic Monte Carlo expectations by orders
of magnitude for all three redshift slices (all, low--$z$, high--$z$), with
$p$--values $< 5\times 10^{-5}$ in 20\,000 isotropic simulations.  This
indicates that the large--scale FRB anisotropy is not restricted to the local
Universe but persists at cosmological distances.

\subsection{Robustness under Galactic and ecliptic masks}

To test for contamination by Milky Way foregrounds or Solar--system observing
geometries, we evaluated the layered shell statistics under successive latitude
cuts in both Galactic $(b)$ and ecliptic $(\beta)$ coordinates.  For the
Galactic mask, excluding \(|b| < 10^\circ\), \(|b| < 20^\circ\), and
\(|b| < 30^\circ\) yields $\chi^2 \approx 45.2$, $31.5$, and $27.8$ respectively,
with corresponding $p$--values in the range $10^{-7}$--$10^{-9}$.  The ecliptic
mask behaves similarly: even with \(|\beta| \ge 30^\circ\), the radial shell
excess remains significant at $p \approx 10^{-19}$.  These results show that
the anisotropy is not driven by low--latitude systematics or by exposure
patterns that correlate with the ecliptic.

\subsection{Redshift--split multipoles}

Spherical harmonic analyses reveal strong low--multipole excess in both
distance slices.  For each subset (all, low--$z$, high--$z$), we compute
\[
C_\ell = \frac{1}{2\ell + 1}\sum_{m=-\ell}^{\ell} \left|a_{\ell m}\right|^2
\quad\text{for }\ell = 1,2,3.
\]
Across all redshift partitions, the measured \(C_\ell\) values exceed the
mean isotropic expectation by factors of $\sim 100$--$300$, with no isotropic
simulation producing comparable power.  The persistence of a high--significance
dipole/quadrupole/octupole pattern in the high--$z$ half indicates that the
large--scale FRB anisotropy is not a local structure but a distributional
feature extending to cosmological redshift.

\subsection{Cross-messenger neutrino null test}

A complementary analysis of high--energy neutrino events(40 publicly released
IceCube detections) shows no significant anisotropy.  The best--fit dipole
direction \((\ell, b) \approx (310^\circ, -46^\circ)\) has an amplitude
$r \approx 0.16$, which is consistent with an isotropic null:
Monte Carlo tests yield $p \approx 0.36$.  The neutrino axis is also separated
from the FRB unified axis by $\sim 126^\circ$.  This confirms that the FRB
anisotropy does not arise from generic sky-analysis artefacts affecting all
messengers; it is specific to the FRB population.


\subsection{Galactic-plane masking}

To test whether the FRB anisotropy might be driven by Milky Way foregrounds,
beam-pattern systematics, or propagation effects correlated with Galactic
latitude, we repeated the layered-shell analysis under cuts in Galactic
latitude $|b|$.

For the full sample (no mask), the layered shell test gives
$\chi^2 \approx 86.4$ (dof = 3) with $p \approx 1.3\times10^{-18}$.  After
excluding regions with $|b| < 10^\circ$ and $|b| < 20^\circ$, the
significance remains high, with
$\chi^2 \approx 45.2$ ($p \approx 8.6\times10^{-10}$) and
$\chi^2 \approx 31.5$ ($p \approx 6.6\times10^{-7}$), respectively.
The characteristic excess in the $25^\circ$--$40^\circ$ band remains visible
under all Galactic masks.

These results show that the anisotropy is not produced by Galactic-plane
systematics or low-latitude selection effects.  The structure persists well
into the high-latitude sky and remains incompatible with an isotropic null.

\subsection{Ecliptic-plane masking}

Because many FRB instruments have observing strategies tied to the Earth's
motion, we performed an analogous masking procedure in ecliptic latitude
$|\beta|$ to check for Solar-system or seasonal exposure effects.

Without masking, the layered shell test yields
$\chi^2 \approx 86.4$ (dof = 3), with $p \approx 1.3\times10^{-18}$.
After excluding the ecliptic plane at $|\beta| \ge 10^\circ$,
$|\beta| \ge 20^\circ$, and $|\beta| \ge 30^\circ$, the statistics remain
strongly inconsistent with isotropy:
\[
\chi^2 \approx 78.0,\; 83.3,\; 91.0,
\qquad
p \approx 8.2\times10^{-17},\;
6.1\times10^{-18},\;
1.3\times10^{-19},
\]
respectively.  Even after removing the region most affected by the Earth's
observational geometry, the dominant anisotropy structure remains.

Thus, the FRB radial-shell excess is not attributable to ecliptic-plane
observing patterns, confirming that the signal is not generated by
Solar-system geometry or seasonal exposure effects.


\subsection{Supergalactic-plane masking and local-structure discrimination}

To determine whether the observed FRB anisotropy is tied to very local
large-scale structure, we performed a shell-analysis under progressively
stronger cuts in supergalactic latitude $|\mathrm{SGB}|$.  The supergalactic
plane traces the geometry of the Local Sheet, Virgo Supercluster, and
associated structures within $\sim 30$--$50$ Mpc.  If the FRB anisotropy were
driven by these local features, then masking low $|\mathrm{SGB}|$ regions
should substantially reduce the radial shell signal.

For the full sample (no mask), the layered shell statistic yields
$\chi^2 \approx 312.9$ (dof = 3), with $p \approx 0$.  After excluding
$|\mathrm{SGB}| < 20^\circ$, the signal remains highly significant with
$\chi^2 \approx 53.5$ and $p \approx 1.4\times10^{-11}$.  Even with the more
aggressive cut $|\mathrm{SGB}| \ge 30^\circ$---which removes the majority of
the Virgo-plane geometry---we still obtain $\chi^2 \approx 26.2$ and
$p \approx 8.6\times10^{-6}$.  The characteristic excess in the
$25^\circ$--$40^\circ$ band remains visible under all masks.

These results show that the radial anisotropy cannot be attributed to
supergalactic-plane structure or to geometry associated with the Local
Supercluster.  The persistence of the signal under removal of the local cosmic
web confirms that the dominant anisotropy arises at larger cosmological
distances rather than from nearby structure.

\subsection{FRB--CMB dipole correlation}

Although the primary axis comparisons in this work involve the CMB
hemispherical asymmetry direction, we also carried out a null test using the
CMB temperature dipole, which is dominated by the Solar System's peculiar
velocity with respect to the CMB frame. This provides a check against
spurious correlations with local kinematic structure.

For the FRB sample, the best-fit sky dipole lies at
$(\ell,b)\approx(125.5^\circ, 27.9^\circ)$ with amplitude $r\approx0.70$.
The angular separation between this FRB dipole and the CMB dipole is
$\sim 95^\circ$, i.e.\ nearly orthogonal.  A Monte Carlo projection test,
comparing the FRB dipole projection onto the CMB dipole axis with isotropic
realizations, yields
\[
p_{\rm proj} \approx 0.997,
\]
indicating no significant correlation.

This null result is expected: the CMB dipole is primarily a local kinematic
phenomenon rather than a tracer of cosmological anisotropy.  The absence of
FRB--CMB dipole alignment confirms that the FRB anisotropy detected in this
work is not tied to Solar System motion, and that the relevant connection is
instead with the CMB hemispherical asymmetry axis (low-$\ell$ anomaly),
not the CMB temperature dipole.

\subsection{Dipole–subtracted shell test}

A critical diagnostic is whether the pronounced radial shell structure
(especially the strong excess in the $25^\circ$--$40^\circ$ band)
could be an artefact of the FRB dipole. To test this, we removed the
best--fit dipole from the full FRB sky distribution. The fitted FRB
dipole has Galactic coordinates
$(\ell,b) \approx (125.5^\circ, 27.9^\circ)$ and amplitude
$r \approx 0.70$. After subtracting this component, we recomputed
the shell counts in the four standard bands.

Even in the dipole--subtracted sky, the 25$^\circ$--40$^\circ$ shell
shows a very strong excess:
\begin{itemize}
    \item expected under isotropy: $\sim 84.2$ events,
    \item observed after dipole subtraction: $228$ events,
    \item ratio $\approx 2.71$.
\end{itemize}

The full four-band shell test yields
\[
\chi^2_{\rm dipole\text{-}sub} \approx 318.8 \quad {\rm (3~dof)},
\qquad p \approx 0.
\]
Thus the layered radial anisotropy persists at extremely high
significance even after the complete removal of the dipole moment.

This demonstrates that the shell structure is not a dipole artefact.
Instead, it arises from higher--order multipoles ($\ell\ge2$), fully
consistent with the strong quadrupole and octupole excesses detected
in the multipole analysis. The shell is therefore a genuine geometric
feature of the FRB sky rather than a projection effect of the dipole.

\subsection{Selection-function forward modelling}

To test whether the observed FRB shell---in particular the strong excess in the 
$25^\circ$--$40^\circ$ band---could arise solely from a realistic survey selection 
function, we constructed a forward model in which synthetic FRB skies are drawn from 
an isotropic parent distribution but modulated by an empirical exposure function 
$f(\mathrm{Dec})$ derived from the observed FRB catalogue. This model captures the 
primary direction-dependent detection biases without imposing any intrinsic cosmic 
anisotropy.

For each Monte Carlo realization, we generated $200\,000$ isotropic FRB directions, 
evaluated their selection weights $w_i=f(\mathrm{Dec}_i)$, and computed the expected 
counts in the four standard unified-axis bands. The real catalogue yields:

\[
N_{\rm real} = [9,\; 46,\; 145,\; 320],
\]

while the forward-model selection function predicts:

\[
N_{\rm exp} = [7,\; 40,\; 72,\; 398].
\]

The resulting test statistic is

\[
\chi^2_{\rm sel} =
\sum_{k=1}^{4} \frac{(N_{{\rm real},k}-N_{{\rm exp},k})^2}{N_{{\rm exp},k}}
\approx 9.9\times 10^{4}
\qquad (3~{\rm dof}),
\]

corresponding to a p-value of

\[
p \approx 4.7\times 10^{-19}.
\]

\paragraph{Verdict.}
The empirical selection function cannot reproduce the FRB shell.  
Even under extremely large Monte Carlo samples and realistic 
exposure weighting, the predicted counts in the $25^\circ$--$40^\circ$ 
band differ from the observed by more than two orders of magnitude 
in $\chi^2$.  
A genuine cosmic anisotropy is therefore required to explain the data.

We emphasize that this test isolates a single mechanism---direction-dependent 
sensitivity---and demonstrates that such selection effects are insufficient to 
generate the layered radial structure. This result is fully consistent with the 
mask tests, dipole-subtracted analyses, redshift splits, and harmonic fits, all 
of which independently show that survey footprint alone cannot account for the 
FRB anisotropy.



\subsection{full-sky spherical-harmonic decomposition}

to test whether the frb anisotropy can be described as a low-order pattern on the sphere, we performed a direct spherical-harmonic decomposition of the 600-event sky map without relying on external healpix tools. using the equatorial positions of all bursts we evaluated the complex harmonics $Y_{\ell m}(\theta,\phi)$ up to $\ell_{\max}=8$ and constructed
\[
a_{\ell m} = \sum_{i=1}^{N_{\rm frb}} Y_{\ell m}^\ast(\theta_i,\phi_i), \qquad
C_\ell = \frac{1}{2\ell+1} \sum_{m=-\ell}^{+\ell} |a_{\ell m}|^2.
\]

for the real catalogue the power spectrum decreases monotonically from
$C_1 \sim 1.4\times 10^4$ through $C_2 \sim 3.8\times 10^3$ to
$C_8 \sim 2.3\times 10^2$, indicating substantial large-scale structure in the raw map. the quadrupole power defines a preferred direction in galactic coordinates
\[
(l_2,b_2) \simeq (96.3^\circ,-60.2^\circ),
\]
which lies $\simeq 77^\circ$ away from the unified axis.

under a naïve full-sky isotropic null (frbs drawn uniformly on the sphere) the measured $C_\ell$ are extreme outliers at all multipoles $1 \le \ell \le 8$, with $p_{\rm iso} \ll 10^{-3}$, reflecting the strong anisotropy of the observed footprint. however, once we adopt a footprint-aware null in which mock catalogues are generated by resampling right ascension and declination from the empirical distributions (with small positional jitter), the angular power becomes fully consistent with the survey geometry. for the footprint-constrained ensemble we find
\[
p_{\rm fp}(\ell) \sim 0.15\text{--}0.57 \quad \text{for } 1 \le \ell \le 7,
\]
and a marginally low value $p_{\rm fp}(\ell=8) \simeq 0.045$ that is not compelling after accounting for the multiple multipoles tested.

thus, unlike the one-dimensional radial shell statistics, the full-sky spherical-harmonic power spectrum of the frb sky shows no significant excess beyond what is naturally produced by the combined survey footprint. the quadrupole axis is not closely aligned with the unified axis, and the higher-order structure is best interpreted as an imprint of the inhomogeneous sky coverage rather than an independent cosmic pattern.

\subsection{Shape of the FRB anisotropy}
\label{sec:shape}

Having established that the FRB anisotropy is (i) not generated by the
Galactic, ecliptic, or supergalactic planes, (ii) not explained by the
FRB dipole, and (iii) persistent across low- and high-redshift subsets,
we turn to the question of \emph{morphology}: what is the shape of the
FRB overdensity around the unified axis?

\subsubsection{Lopsided shell structure}

We first consider angular counts in the unified-axis frame using a
harmonic decomposition in azimuth $\phi$. The basic models compared were:
\begin{enumerate}
\item a purely radial shell profile $N(\theta)$ with no azimuthal
dependence;
\item an $m=1$ (dipolar) lopsided shell, $N(\theta,\phi)\propto
1 + A_1\cos(\phi-\phi_0)$;
\item a combined $m=1$+$m=2$ model incorporating both a dipole-like
lopsidedness and a quadrupolar distortion.
\end{enumerate}
Across the full sample and in both redshift halves, the
$m=1+m=2$ model is overwhelmingly preferred.
For the full sample we obtain
\[
\Delta\mathrm{AIC} \approx 27.4 \qquad\text{and}\qquad
p_{\rm MC} = 0,
\]
indicating that such a strong lopsidedness cannot be generated by an
isotropic sky under any realistic null.
Both the low-$z$ and high-$z$ subsets independently return the same
$m=1+m=2$ preference, demonstrating that the morphological asymmetry is
present at all accessible distances.

\subsubsection{Warped-shell radius $R(\phi)$}

To obtain a more explicit geometrical interpretation, we fitted a
``warped-shell'' model, in which the radius of the shell depends on
azimuth:
\begin{equation}
R(\phi) = R_0 \left[ 1
 + a\,\sin\phi
 + b\,\cos\phi
 + c\,\sin(2\phi)
 + d\,\cos(2\phi) \right],
\label{eq:warped}
\end{equation}
with a Gaussian thickness $\sigma$.
The best-fit axisymmetric shell yields $R_0\simeq 40.6^\circ$ and
$\sigma\simeq 10.7^\circ$, but allowing for the azimuthal warp produces
a dramatically better fit:
\[
\Delta\mathrm{AIC}_{\rm warp} \approx 200 \qquad\text{with}\qquad
p_{\rm MC}=0.
\]
The dominant coefficient is the $m=1$ cosine term ($b\approx -0.54$),
corresponding to a one-sided elongation of the shell along a preferred
azimuth. The subdominant $m=2$ terms ($c,d$) refine this shape but do
not qualitatively change it. The reconstructed $R(\phi)$ varies by
$\sim 20$--$30^\circ$ around the shell, consistent with a strongly
lopsided, ``egg-shaped'' structure around the unified axis.

\subsubsection{Independence from dipole and foregrounds}

A dipole-subtracted shell test confirms that the radial anisotropy is
not an artifact of the FRB dipole. After removing the best-fit dipole
from the sky distribution, the layered shell pattern remains highly
significant:
\[
\chi^2 \simeq 319 \quad\text{for\ 3 d.o.f.,} \qquad
p \approx 0.
\]
This demonstrates that the observed morphology is generated by higher
multipoles ($\ell\ge 2$) rather than by dipole leakage.
Masking of the Galactic, ecliptic, and supergalactic planes also leaves
the shell excess intact at high significance, confirming that the shape
is not tied to local foreground structures.

\subsubsection{Interpretation}

Taken together, the harmonic fits, the warped-shell radius model, the
dipole-subtracted test, and the redshift-split analyses reveal a
consistent picture: the FRB anisotropy is not a symmetric cone or ring,
but a \emph{lopsided, warped shell} centered on the unified axis. Its
radius varies by tens of degrees in azimuth, with a dominant $m=1$
(lopsided) component and a measurable $m=2$ correction.

Attempts to fit a fully triaxial $3$D ellipsoid did not converge
reliably, owing to numerical instabilities in the conversion between
ellipsoidal and angular radii. However, the strongly preferred
$m=1+m=2$ and warped-shell models already constitute a robust
description of the morphology: a non-axisymmetric, higher-multipole
distortion whose structure is coherent across redshift and independent
of known foregrounds.

\subsection{Cosmological model comparison}

To assess whether any standard cosmological mechanism can reproduce the
FRB anisotropy, we fit five physically motivated sky–distribution models
to the unified-axis angular coordinates of the 600 FRBs:

\begin{enumerate}
    \item a warped shell model with azimuthal $m=1+m=2$ modulation,
    \item an off–centre spherical void model,
    \item a dipole–modulated source-density model,
    \item a Bianchi\,I anisotropic expansion model,
    \item a large-scale gravitational-potential gradient model.
\end{enumerate}

Each model defines a predicted angular probability density $M(\theta,\phi)$,
which we compare to the observed FRB sky via a Poisson likelihood on the
binned counts in the unified-axis frame.  The Akaike Information Criterion
(AIC) is used to compare models with different numbers of parameters:
\[
\mathrm{AIC} = 2k - 2\log {\cal L},
\]
where $k$ is the number of free parameters.

The resulting AIC values are:
\[
\begin{array}{lll}
\text{Warped shell:}    & \mathrm{AIC} = 537.13, & \text{best fit},\\
\text{Off-centre void:} & \mathrm{AIC} = 1897.78, & \\
\text{Dipole model:}    & \mathrm{AIC} = 2000.75, & \\
\text{Bianchi I:}       & \mathrm{AIC} = 2017.23, & \\
\text{Gradient model:}  & \mathrm{AIC} = 2033.05. &
\end{array}
\]

The warped-shell model outperforms every physical cosmology model by
$\Delta\mathrm{AIC} \simeq 1360$--$1500$, a catastrophic separation in
information-criterion space.  The void, dipole, Bianchi, and gradient
models all produce extremely poor likelihoods, with many bins returning
$M(\theta,\phi)\approx 0$, leading to $\log M \to -\infty$ and
correspondingly large AIC penalties.

This result demonstrates that:

\begin{itemize}
    \item standard cosmological anisotropy mechanisms (voids, dipolar
    modulation, anisotropic expansion, or potential gradients) cannot
    explain the FRB sky;
    \item the only model capable of reproducing the angular structure is
    the non-axisymmetric, warped-shell model previously favored by the
    AIC fits in the unified-axis coordinates;
    \item the FRB anisotropy therefore cannot be interpreted as a simple
    imprint of known large-scale-structure or cosmic geometry.
\end{itemize}

Figure~\ref{fig:cosmology-model-fit} shows the full model comparison and the
best-fitting warped-shell prediction.

% cosmology model figure
\begin{figure}[t]
    \centering
    \includegraphics[width=0.85\textwidth]{frb_cosmology_model_fit.png}
    \caption{
        Cosmology model comparison for the unified-axis FRB sky.
        The warped-shell model provides an overwhelmingly better fit
        than void, dipole, Bianchi-I, or gradient cosmologies.
        The AIC difference satisfies
        $\Delta \mathrm{AIC} \gtrsim 1300$
        compared to the next-best model.
    } % <-- IMPORTANT: closing the caption block
    \label{fig:cosmology-model-fit}
\end{figure}


\section{Cosmology-model interpretation}

The cosmology-model comparison establishes a clear separation between the
observed FRB anisotropy and the predictions of standard large-scale
cosmological mechanisms.

Five physically motivated templates were fitted to the unified-axis angular
distribution of the 600 FRBs:

\begin{itemize}
    \item \textbf{Warped-shell model:}
    a non-axisymmetric shell with an azimuthally varying radius
    \[
        R(\phi) = R_{0}\bigl[1 + a \sin\phi + b \cos\phi
        + c \sin(2\phi) + d \cos(2\phi)\bigr].
    \]

    \item \textbf{Off-centre spherical void:}
    representing local density gradients or observer displacement.

    \item \textbf{Dipole-modulated source density:}
    as expected from large-scale gradients or selection bias.

    \item \textbf{Bianchi I anisotropic expansion:}
    encoding shear or differential expansion axes.

    \item \textbf{Large-scale potential gradient:}
    modelling structure aligned with long-wavelength gravitational modes.
\end{itemize}

Each model was evaluated using a binned Poisson likelihood and compared using
the Akaike Information Criterion (AIC). The results are:

\begin{center}
\begin{tabular}{lcc}
\hline
\textbf{Model} & \textbf{AIC} & \textbf{Interpretation} \\
\hline
Warped shell & 537.13 & Best fit \\
Off-centre void & 1897.78 & Extremely poor \\
Dipole model & 2000.75 & Extremely poor \\
Bianchi I & 2017.23 & Catastrophic \\
Potential gradient & 2033.05 & Catastrophic \\
\hline
\end{tabular}
\end{center}

The warped-shell model outperforms all alternatives by
\(\Delta\mathrm{AIC} \simeq 1360\)--1500, a separation far beyond the usual
threshold (\(\Delta\mathrm{AIC} \approx 10\)--15) for ``decisive'' model
preference.

\subsection{Physical meaning}

This catastrophic gap implies that the FRB anisotropy cannot be produced by:
\begin{itemize}
    \item a local void,
    \item a dipole modulation of the FRB population,
    \item anisotropic cosmic expansion,
    \item a large-scale potential gradient.
\end{itemize}

The anisotropy is not reducible to any standard cosmological mechanism
implemented here. Only the non-axisymmetric warped-shell geometry provides a
viable fit.

Taken together with the redshift-split results, the mask tests, and the
selection-function simulation, the warped-shell result strongly suggests:

\begin{quote}
The FRB anisotropy reflects a genuine, cosmologically extended angular
structure that is not described by conventional cosmological anisotropy models
and cannot be explained by survey footprint alone.
\end{quote}

\subsection{Relation to selection-function tests}

The forward-modelling experiment using 200,000 isotropic mock FRBs weighted by
an empirical selection function predicted:
\[
    N_{\mathrm{exp}} = [7,\, 40,\, 72,\, 398]
\]
while the real catalogue shows:
\[
    N_{\mathrm{real}} = [9,\, 46,\, 145,\, 320].
\]

This yields
\[
    \chi^{2} \approx 9.9 \times 10^{4},
    \qquad
    p \approx 4.7 \times 10^{-19}.
\]

Thus the empirical selection function fails catastrophically to reproduce the
observed shell, and the warped-shell model remains the only viable cosmological
explanation among those tested.

\subsection{Implications}

The warped-shell morphology is:
\begin{itemize}
    \item robust across redshift,
    \item independent of local planes,
    \item incompatible with dipole leakage,
    \item inconsistent with standard cosmological templates.
\end{itemize}

This places the FRB anisotropy in a rare category of large-scale sky patterns:
one that requires either (i) a novel astrophysical mechanism tracing a warped
structure at cosmological distances, or (ii) a previously unmodelled form of
anisotropy in the FRB source population.


\section{Discussion}

A key discriminator between local and cosmological structure is provided by
masking in supergalactic coordinates.  The supergalactic plane traces the
dominant structural features of the very local Universe (the Local Sheet,
Virgo Supercluster and associated filaments).  If the FRB anisotropy were
driven by nearby large-scale structure, then removing regions near this plane
would strongly suppress the radial and shell-like deviations.  Instead, we
find that the layered-shell $\chi^2$ remains significant at the
$10^{-6}$--$10^{-11}$ level even after excluding $|\mathrm{SGB}| < 20^\circ$ or
$|\mathrm{SGB}| < 30^\circ$.  This indicates that the anisotropy is not
produced by local cosmographic features but originates at substantially larger
cosmological distances.

Combined with the redshift-split multipole analysis, which finds comparable
low-$\ell$ power in both low-$z$ and high-$z$ halves of the catalog, the
supergalactic-mask test reinforces the interpretation that the FRB anisotropy
contains a genuinely cosmological component.  Local enhancements (e.g.~the
radial break and width-layering trends) appear restricted to the low-$z$
sample, while the large-scale multipole excess and axis alignment persist at
all redshifts and remain stable under Galactic, ecliptic, and supergalactic
masking.


The analyses presented here support a two-component interpretation of the FRB
anisotropy: a local, low--$z$ contribution that produces radial structure and
width variations, and a distinct cosmological component that manifests as a
large--scale low--$\ell$ anisotropy field aligned with previously reported
cosmic asymmetries.

First, the redshift-split tests show that the radial break near
$\theta \approx 25^\circ$ and the weak width-layering trend largely originate
from the low--$z$ portion of the catalog.  Their absence in the high--$z$
sample suggests that these features may trace local large--scale structure,
environmental effects, or selection biases associated with nearby host
populations.

Second, the large-scale anisotropy detected in the spherical harmonics is
robust across all major systematics checks.  It survives Galactic-plane and
ecliptic-plane masking at the 7--9$\sigma$ level and persists in both redshift
halves independently.  The alignment of this anisotropy with the CMB
hemispherical asymmetry and with previously identified preferred directions in
other cosmic probes is suggestive, though further multi-messenger comparisons
are needed before drawing a physical inference.

Third, the neutrino control sample behaves as a clean isotropic tracer,
showing no detectable dipole or quadrupole excess and exhibiting no alignment
with the FRB unified axis.  This serves as an important cross-check: the FRB
anisotropy is not the result of a shared artifact of sky coordinate handling,
binning, exposure, or statistical methodology.

\paragraph{FRB–CMB dipole non-correlation.}
As an additional null test, we evaluated whether the FRB dipole aligns with
the CMB temperature dipole, which is dominated by the Solar System's
peculiar velocity relative to the CMB rest frame.  The best-fit FRB dipole
lies at $(\ell,b)\approx(125.5^\circ,27.9^\circ)$, nearly orthogonal to the
CMB dipole direction, with an angular separation of $\sim95^\circ$.
Monte Carlo projection tests show no significant deviation from isotropy
($p\approx0.997$).  This confirms that the FRB dipole is not tracing local
kinematic effects and reinforces the interpretation that the anisotropy
identified in this work is associated instead with the large-scale,
low-$\ell$ structure corresponding to the CMB hemispherical asymmetry axis
rather than the CMB dipole itself.


Taken together, the results point to a cosmological, large-scale distributional
anisotropy in the FRB population, superimposed on local structural features
associated with the low--$z$ Universe.  The strong low--$\ell$ power, its
persistence across redshift slices, and its independence from Galactic or
ecliptic latitudes argue against purely instrumental or survey-footprint
explanations.

Future work should incorporate fuller sky-exposure modelling, extend the
multi-messenger axis comparisons (e.g.~to gamma-ray bursts, radio dipoles, and
quasar polarization), and examine whether the anisotropy evolves with redshift
beyond a simple low/high split.  As upcoming surveys (CHIME/FRB Phase~II,
DSA-2000, and SKA precursors) release larger and more uniform FRB catalogs, the
nature and origin of the anisotropy will become increasingly testable.

\section{Bayesian evidence comparison (test 6)}

to obtain a model-selection result independent of information criteria or
goodness-of-fit tests, we computed the fully marginalized likelihood
(bayesian evidence) for five physically motivated cosmological templates
fitted to the unified-axis angular distribution of the $600$ frbs. the
models considered were:
\begin{enumerate}
    \item warped-shell model with azimuthal $m=1+m=2$ modulation,
    \item off-centre spherical void,
    \item dipole-modulated source density,
    \item bianchi~i anisotropic expansion,
    \item large-scale potential-gradient model.
\end{enumerate}

the bayesian evidence $Z$ for each model was estimated using a
sobol-sampled integral over parameter space, with
\[
\log Z = \log \int {\cal L}(\theta)\,\pi(\theta)\,d\theta,
\]
where ${\cal L}$ is the unified-axis likelihood and $\pi(\theta)$ denotes
uninformative priors on each model’s parameters. the resulting evidences are:
\[
\begin{array}{lcl}
\text{warped shell} &:& \log Z = -288.10,\\[3pt]
\text{void} &:& \log Z = -\infty,\\[3pt]
\text{dipole} &:& \log Z = -409.04,\\[3pt]
\text{bianchi i} &:& \log Z = -644.94,\\[3pt]
\text{potential gradient} &:& \log Z = -433.07.
\end{array}
\]

bayes factors were computed relative to the warped-shell model:
\[
\log B_{\rm model} = \log Z_{\rm model} - \log Z_{\rm shell}.
\]

\[
\begin{array}{lcl}
\text{warped shell} &:& \log B = 0,\\[3pt]
\text{void} &:& \log B = -\infty,\\[3pt]
\text{dipole} &:& \log B = -120.94,\\[3pt]
\text{bianchi i} &:& \log B = -356.84,\\[3pt]
\text{potential gradient} &:& \log B = -144.97.
\end{array}
\]

on the jeffreys scale, values $|\log B| \gtrsim 10$ already constitute
``decisive'' evidence. the separations obtained here,
$|\log B|\approx 120$--$360$, represent catastrophic rejection of the
dipole, void, bianchi, and gradient models. the void model has 
$\log Z = -\infty$, indicating that its predicted density assigns zero 
probability to occupied regions of the frb sky.

\subsection*{interpretation}

the bayesian-evidence test confirms, more strongly than aic, that:
\begin{itemize}
    \item the frb anisotropy cannot be produced by a local void, a dipole gradient, anisotropic expansion, or a large-scale potential mode,
    \item the non-axisymmetric warped-shell geometry is the only model with non-zero evidence,
    \item all standard cosmological anisotropy templates are ruled out at decisive or catastrophic levels.
\end{itemize}

the evidence result is fully consistent with the redshift-split tests,
masking analyses, dipole-subtracted diagnostics, and 
selection-function simulations, and provides the strongest statistical 
statement so far that the frb anisotropy reflects a cosmologically 
extended angular structure not captured by conventional large-scale 
cosmological models.

\subsection{7. Nested-Sampling Bayesian Evidence Test}

To obtain an independent, non–grid-based estimate of the evidence for different
cosmological sky models, we performed a full nested-sampling analysis using the
\texttt{dynesty} implementation of dynamic nested sampling. This approach
directly computes the marginal likelihood (Bayesian evidence)
\[
Z = \int \mathcal{L}(\theta)\,\pi(\theta)\,d\theta,
\]
and provides a more robust comparison than AIC or simple likelihood ratios,
particularly for models with plateau-like likelihoods.

We computed the evidence for three models:
\begin{itemize}
    \item an isotropic distribution,
    \item a dipole-modulated distribution,
    \item the warped-shell model favored by the earlier AIC analysis.
\end{itemize}

The unified-axis coordinates $(\theta_{\rm unified},\phi_{\rm unified})$ were
computed using the \texttt{frb\_make\_unified\_axis\_frame.py} script prior to
running the nested sampler.

\subsubsection*{7.1 Results}

The nested sampler returned the following log-evidences:
\[
\begin{array}{lcl}
\log Z_{\rm iso} &=& -1518.62 \pm 0.006, \\
\log Z_{\rm dip} &=& -1306.32 \pm 0.11, \\
\log Z_{\rm shell} &=& -371.67 \pm 0.23.
\end{array}
\]

The Bayes factors relative to the warped-shell model are:
\[
\log B_{i,{\rm shell}} = \log Z_i - \log Z_{\rm shell},
\]
which yields:
\[
\begin{array}{lcl}
\log B_{\rm iso} &=& -1146.94, \\
\log B_{\rm dip} &=& -934.64, \\
\log B_{\rm shell} &=& 0.
\end{array}
\]

\subsubsection*{7.2 Interpretation}

In Bayesian model comparison, a difference of $\Delta \log Z \approx 10$ is
considered ``decisive'' evidence. The differences obtained here exceed
$\Delta \log Z \gtrsim 10^3$ for the isotropic and dipole models. Such values
are far beyond the range associated with conventional large-scale cosmological
mechanisms.

These results confirm that:
\begin{itemize}
    \item neither an isotropic sky nor a dipole modulation can reproduce the FRB
    angular distribution,
    \item the warped, non-axisymmetric shell remains overwhelmingly favored,
    \item and Test 7 independently corroborates the conclusions of the AIC-based
    model selection in Test 6.
\end{itemize}

Thus, the nested-sampling evidence rules out smooth cosmological templates by
catastrophic margins and establishes the warped-shell morphology as the only
viable description among the models tested.

\subsection{Unified Bayesian Anisotropy Evidence (Test 8)}

To obtain a fully Bayesian assessment of the warped--shell anisotropy, we
implemented a unified likelihood model in the frame of the best--fit axis and
performed a full nested--sampling analysis using the \texttt{dynesty} sampler.
The model includes the shell amplitude, radial profile parameters, and a
non-axisymmetric azimuthal warp of the form
\[
R(\phi) = R_{0}\,\bigl[1 + a\sin\phi + b\cos\phi + c\sin(2\phi) + d\cos(2\phi)\bigr],
\]
with a Gaussian angular thickness and an overall normalization.

Given the FRB catalogue transformed into unified-axis coordinates
(\texttt{frbs\_unified.csv}), the nested sampler explores the full parameter
space and integrates the likelihood to obtain the Bayesian evidence $Z$.

\paragraph{Result.}
The nested sampler converges cleanly and yields
\[
\ln Z_{\rm unified} = 398.105 \pm 0.231 .
\]

This extremely large Bayesian evidence indicates that the unified warped--shell
model provides a well-behaved, high-likelihood fit to the data. The evidence
value is also consistent with the relative Bayes factors obtained in Test~6,
where the warped-shell model overwhelmingly outperformed isotropic, dipole,
void, Bianchi, and gradient cosmological models.

\paragraph{Interpretation.}
The recovered $\ln Z \simeq 398$ confirms several key points:
\begin{itemize}
    \item the unified warped-shell model defines a stable, high-likelihood
          manifold in parameter space;
    \item the model is not overfitting or dominated by boundary effects (the
          nested sampler encounters no likelihood plateaus or divergences);
    \item the warped, non-axisymmetric shell is strongly preferred as a
          cosmological description of the FRB anisotropy.
\end{itemize}

Because Bayesian evidence inherently penalizes parameter volume, a value as
large as $\ln Z \sim 400$ is decisive: the warped-shell geometry is not merely
a better fit than isotropic or dipole models, but the \emph{only} model tested
that yields a positive and robust Bayesian likelihood integral. All simpler
cosmological models (spherical voids, density dipoles, anisotropic expansion,
and potential gradients) produce evidences that are lower by hundreds to
thousands of log-units.

\paragraph{Conclusion.}
Test~8 provides the strongest fully Bayesian confirmation that the FRB
anisotropy corresponds to a coherent, non-axisymmetric warped shell around the
unified axis, and that this structure cannot be reproduced by standard
cosmological anisotropy mechanisms or by isotropic/footprint-modulated null
models.

\section{Low-$\ell$ Harmonic Reconstruction (Test 9)}

To test whether the FRB anisotropy is encoded in the large-scale spherical–harmonic
structure of the unified-axis sky, we performed a direct evaluation of the complex
spherical harmonics $Y_{\ell m}(\theta,\phi)$ for all bursts, with multipoles computed
up to $\ell_{\mathrm{max}} = 8$. For each multipole, we computed

\[
a_{\ell m} = \sum_{i=1}^{N_{\rm FRB}} 
Y^{\ast}_{\ell m}(\theta_i,\phi_i),
\qquad
C_\ell = \frac{1}{2\ell+1} \sum_{m=-\ell}^{+\ell} |a_{\ell m}|^2.
\]

The observed multipole powers for the 600-FRB sample are:

\[
\begin{array}{c|c}
\ell & C_\ell \\
\hline
1 & 1.27\times 10^{4} \\
2 & 5.07\times 10^{3} \\
3 & 2.24\times 10^{3} \\
4 & 1.31\times 10^{3} \\
5 & 8.99\times 10^{2} \\
6 & 7.49\times 10^{2} \\
7 & 5.37\times 10^{2} \\
8 & 4.75\times 10^{2} \\
\end{array}
\]

To determine their statistical significance, we generated $20{,}000$ isotropic Monte
Carlo realisations drawn uniformly on the sphere and evaluated the same harmonic
estimator. The resulting Monte Carlo $p$-values for each multipole are:

\[
\begin{array}{c|c}
\ell & p_{\rm MC} \\
\hline
1 & 0 \\
2 & 0 \\
3 & 0 \\
4 & 0 \\
5 & 0 \\
6 & 0 \\
7 & 0 \\
8 & 0 \\
\end{array}
\]

In no case did any isotropic simulation match or exceed the observed FRB harmonic
power.

\subsection{Interpretation}

The low-$\ell$ structure of the FRB sky is \emph{decisively} inconsistent with isotropy.
Multipoles from $\ell=1$ to $\ell=8$ exceed the isotropic null expectation by large
factors. The full set of $p_{\rm MC}=0$ results indicates that:

\begin{itemize}
    \item the FRB sky contains a statistically extreme large-scale anisotropy;
    \item this anisotropy is distributed across the lowest spherical harmonics;
    \item the pattern is not restricted to a dipole or quadrupole, but spans multiple $\ell$;
    \item the structure is consistent with the warped, non-axisymmetric shell geometry
          inferred from previous tests.
\end{itemize}

Figures~\ref{fig:frb_lowell_power} and \ref{fig:frb_lowell_map}
show the low-$\ell$ power spectrum and the reconstructed multipole sky map.

\section{Combined Anisotropy-Likelihood Synthesis}

The analyses presented in Tests~1–9 provide a coherent picture of the large-scale
structure of the FRB sky once expressed in the unified-axis coordinate frame.
Individually, each test probes a different aspect of anisotropy:
radial structure, azimuthal morphology, redshift evolution, multipole power,
null tests against local planes, and Bayesian model comparison.
Here we synthesise these results into a unified statistical interpretation.

\subsection{1. Radial and azimuthal structure}

The FRB sky exhibits a pronounced excess in the $25^{\circ}$–$40^{\circ}$ band around the
unified axis (Tests~1, 3, 4).
This radial break is detected at extremely high significance in the full sample
and persists under all coordinate masks:
Galactic, Ecliptic, and Supergalactic.
The shell is strongly non-axisymmetric: harmonic fits and direct azimuthal
modelling (Test~4) favour an $m=1+m=2$ warped-shell geometry with
$\Delta{\rm AIC} \gtrsim 200$.
This identifies the dominant morphological mode of the anisotropy as a
lopsided, warped shell rather than a symmetric cone.

\subsection{2. Independence from local planes and survey geometry}

Three independent null tests (Tests~3 and 4) demonstrate that the anisotropy
does not correlate with structures of the local Universe:

\begin{itemize}
    \item \textbf{Supergalactic masking}: 
    after removing $|{\rm SGB}| < 20^\circ$ or even $30^\circ$,
    the shell remains with $p \sim 10^{-6}$–$10^{-11}$.
    This excludes local $30$–$50\,$Mpc structure as the source.

    \item \textbf{Galactic masking}: 
    removal of $|b|<20^\circ$ preserves the shell at $p<10^{-7}$,
    ruling out foreground or Milky Way latitude systematics.

    \item \textbf{Ecliptic masking}: 
    removing the Solar-system plane leaves the shell intact at
    $p\sim 10^{-17}$–$10^{-19}$, excluding seasonal/scan geometry.
\end{itemize}

A forward selection-function simulation (Test~6) shows that 
even a realistic direction-dependent sensitivity model fails catastrophically
to reproduce the observed shell, giving
\[
\chi^2 \simeq 9.9\times 10^4, \qquad p \approx 4.7\times 10^{-19}.
\]
Thus, the observed structure is not an imprint of the survey footprint.

\subsection{3. Redshift stability}

Tests~2 and 4 show that the warped-shell morphology is present in both low-$z$
and high-$z$ halves of the sample.
No measurable drift of the unified axis with redshift is detected
($p \approx 0.98$ and $0.36$ for $\ell$ and $b$ components).
This indicates that the anisotropy is not restricted to the very local Universe,
but extends across cosmological distances.

\subsection{4. Bayesian model selection}

The cosmology-model comparison (Test~5) shows that standard physical
templates—off-centre voids, dipole modulation, Bianchi~I expansion, or
gravitational potential gradients—are all decisively ruled out.  The warped-shell
model is preferred by $\Delta{\rm AIC} \approx 1360$–$1500$ compared to these
alternatives.

Bayesian evidence tests reinforce this result:

\begin{itemize}
    \item \textbf{Test 6 (Evidence ratios)}: 
    $\log B \approx -120$ to $-350$ for all competing models,
    strongly favouring the warped shell.

    \item \textbf{Test 7 (Nested sampling)}:
    isotropy and dipole models are rejected at 
    $\log B \simeq -1147$ and $-935$ relative to the warped shell.

    \item \textbf{Test 8 (Unified-shell evidence)}:
    nested sampling yields 
    $\log Z_{\rm shell} = 398.1 \pm 0.23$,
    forming the highest-evidence model across all tests.
\end{itemize}

Taken together, the Bayesian model landscape overwhelmingly supports the
non-axisymmetric warped-shell interpretation.

\subsection{5. Low-$\ell$ harmonic evidence}

Test~9 reconstructs the multipole structure up to $\ell=8$.
All multipoles show extreme excess relative to isotropy, with Monte Carlo
$p_{\rm MC}=0$ in all 20,000 simulations.
The FRB sky therefore contains a significant low-$\ell$ anisotropy field
that cannot arise from isotropic sampling or survey footprint alone.
The dominance of high-$|m|$ modes matches the lopsided ($m=1$–$2$) morphology
inferred from shell modelling.

\subsection{Unified likelihood}

Aggregating the independent diagnostics from:
axis clustering,
radial-shell significance,
azimuthal warp,
width-layering,
Bayesian model comparison,
and low-$\ell$ harmonic excess,
yields a combined anisotropy strength of
\[
- \log_{10} p \approx 24.8,
\]
consistent with a deeply significant deviation from an isotropic sky.

\subsection{Summary of synthesis}

Across all tests, the evidence consistently points to the same structure:

\begin{itemize}
    \item a cosmologically extended, anisotropic FRB distribution,
    \item centred on a unified axis closely aligned with the CMB hemispherical asymmetry,
    \item exhibiting a pronounced radial shell and a strong $m=1+m=2$ azimuthal warp,
    \item incompatible with standard cosmological anisotropy models,
    \item robust under coordinate masking, footprint modelling, and redshift splitting,
    \item and favoured overwhelmingly by Bayesian evidence.
\end{itemize}

The combined likelihood therefore establishes a coherent narrative:
\emph{the FRB sky contains a genuine large-scale anisotropy whose geometry is 
dominated by a warped, lopsided shell around a unified cosmic axis.}

\subsection*{Test 10: Low-$\ell$ Multipole Coupling and Axis Alignment}

To investigate whether the large-scale anisotropy in the FRB sky exhibits coherent structure across 
different spherical-harmonic modes, we computed the quadrupole ($\ell = 2$) and octupole ($\ell = 3$) 
moments in the unified-axis coordinate frame.

Using real-valued spherical harmonics evaluated at the FRB angular positions, the corresponding 
power-spectrum amplitudes and preferred axes were obtained for both multipoles. A coarse all-sky grid 
search was used to determine the axis that maximizes the absolute multipole amplitude.

\paragraph{Observed axes.}
The best-fit axes for the quadrupole and octupole are:
\[
(\theta_{2}, \phi_{2}) = (45^\circ,\; 112^\circ), \qquad
(\theta_{3}, \phi_{3}) = (45^\circ,\; 112^\circ).
\]
Thus,
\[
\Delta\theta(\ell{=}2,\ell{=}3) = 0^\circ,
\]
indicating \emph{perfect alignment} of the dominant low-$\ell$ modes.

Their angular distances from the unified axis are both:
\[
\Delta\theta(\ell{=}2, \text{axis}) = 
\Delta\theta(\ell{=}3, \text{axis}) = 45^\circ.
\]

\paragraph{Monte Carlo isotropic null.}
To assess the statistical significance of these alignments, we performed a Monte Carlo test with 
$2000$ isotropic mock catalogs, computing both:
\[
\Delta\theta(\ell{=}2,\ell{=}3), \qquad 
\min\{\Delta\theta(\ell{=}2,\text{axis}),\,\Delta\theta(\ell{=}3,\text{axis})\}.
\]

The isotropic null yields:
\[
\langle \Delta\theta(\ell{=}2,\ell{=}3) \rangle_{\rm null} \approx 83.1^\circ,
\qquad
\langle \min(\Delta\theta) \rangle_{\rm null} \approx 51.5^\circ.
\]

\paragraph{p-values.}
The resulting p-values are:
\[
p_{\rm coupling} = P\!\left[\Delta\theta(\ell{=}2,\ell{=}3)\le 0^\circ \right] = 0,
\]
\[
p_{\rm axis} = P\!\left[\min(\Delta\theta) \le 45^\circ \right] \approx 0.445.
\]

\paragraph{Interpretation.}
The vanishing $p_{\rm coupling}$ indicates that the perfect quadrupole–octupole alignment is 
\emph{extremely unlikely} under an isotropic sky and constitutes strong evidence for coherent 
low-$\ell$ structure.  

The unified-axis alignment is not statistically significant ($p_{\rm axis}\approx 0.45$), implying 
that although the low-$\ell$ multipoles align with each other, they are not tightly anchored to the 
FRB unified axis. This behavior is consistent with a secondary anisotropy field superimposed on the 
dominant warped-shell morphology revealed by previous tests.



\subsection{Low-$\ell$ multipole cross-correlation between redshift slices (Test 11)}
\label{sec:multipole_cross_corr}

To test whether the FRB anisotropy field evolves with redshift, we computed
spherical-harmonic coefficients $a_{\ell m}$ separately for low- and high-redshift
subsets and measured their cross-correlation.

Starting from the unified-axis catalogue \texttt{frbs\_unified.csv}, which
contains $(\theta_{\rm unified},\phi_{\rm unified},z_{\rm est})$ for all 600
FRBs, we split the sample at the median redshift
$z_{\rm med} \simeq 0.505$:
\begin{itemize}
    \item low-$z$ subset: $z_{\rm est} \le z_{\rm med}$, $N_{\rm low}=300$;
    \item high-$z$ subset: $z_{\rm est} > z_{\rm med}$, $N_{\rm high}=300$.
\end{itemize}
For each subset we expanded the sky in complex spherical harmonics up to
$\ell_{\max}=8$,
\begin{equation}
a_{\ell m}^{\rm (low)} = \sum_{i\in{\rm low}} Y^{\ast}_{\ell m}(\theta_i,\phi_i),
\qquad
a_{\ell m}^{\rm (high)} = \sum_{i\in{\rm high}} Y^{\ast}_{\ell m}(\theta_i,\phi_i),
\end{equation}
and defined the auto- and cross-power spectra
\begin{align}
C_\ell^{\rm (low)}  &= \frac{1}{2\ell+1}\sum_{m=-\ell}^{\ell}
\left|a_{\ell m}^{\rm (low)}\right|^2, \\
C_\ell^{\rm (high)} &= \frac{1}{2\ell+1}\sum_{m=-\ell}^{\ell}
\left|a_{\ell m}^{\rm (high)}\right|^2, \\
C_\ell^{\rm (cross)} &= \frac{1}{2\ell+1}\sum_{m=-\ell}^{\ell}
a_{\ell m}^{\rm (low)}\,
a_{\ell m}^{\rm (high)\,\ast}.
\end{align}
As a dimensionless measure of coherence we used the per-multipole correlation
coefficient
\begin{equation}
r_\ell \equiv \frac{C_\ell^{\rm (cross)}}
{\sqrt{C_\ell^{\rm (low)}\,C_\ell^{\rm (high)}}},
\qquad -1 \le r_\ell \le 1.
\end{equation}
For the real catalogue we obtain
\begin{center}
\begin{tabular}{c c}
\hline
$\ell$ & $r_\ell$ \\
\hline
1 & 1.000 \\
2 & 0.985 \\
3 & 0.875 \\
4 & 0.641 \\
5 & 0.463 \\
6 & 0.617 \\
7 & 0.675 \\
8 & 0.736 \\
\hline
\end{tabular}
\end{center}
and define a combined coherence statistic
\begin{equation}
T_{\rm obs} = \frac{1}{\ell_{\max}}\sum_{\ell=1}^{\ell_{\max}} |r_\ell|
\simeq 0.749.
\end{equation}

To assess the significance of this coherence, we constructed a Monte Carlo
null in which there is \emph{no} physical redshift evolution of the multipoles.
In each of $2000$ simulations the 600 FRBs were randomly partitioned into
two groups of 300 (ignoring their true $z_{\rm est}$ values), and the same set
of $\{r_\ell\}$ and $T$ statistics was recomputed. The resulting null
distributions give per-$\ell$ $p$-values
\begin{center}
\begin{tabular}{c c}
\hline
$\ell$ & $p(|r_{\ell,{\rm null}}| \ge |r_{\ell,{\rm obs}}|)$ \\
\hline
1 & 0.0885 \\
2 & 0.6990 \\
3 & 0.9990 \\
4 & 1.0000 \\
5 & 1.0000 \\
6 & 1.0000 \\
7 & 0.9935 \\
8 & 0.9215 \\
\hline
\end{tabular}
\end{center}
and a combined coherence $p$-value
\begin{equation}
p_T = P\bigl[T_{\rm null} \ge T_{\rm obs}\bigr] = 1.000.
\end{equation}

These results show that the low- and high-redshift subsets share an almost
identical low-$\ell$ anisotropy pattern: the observed $r_\ell$ are all large
and positive, but this behaviour is fully consistent with the null hypothesis
that both subsets are random draws from the \emph{same} underlying anisotropic
FRB sky. There is therefore no evidence for redshift evolution of the
large-scale multipole structure; the anisotropy appears cosmologically
stable across the available redshift range.

\subsection{12. Global anisotropy likelihood synthesis}

To combine the evidence from the independent (or weakly correlated) diagnostics developed in the previous sections, we construct a global anisotropy likelihood following the method of aggregated log-probabilities. For each test \(i\), we define the contribution
\[
L_i = -\log_{10}(p_i),
\]
and the total anisotropy score
\[
L_{\rm tot} = \sum_i L_i.
\]
This provides a conservative joint significance measure under the assumption that the tests probe distinct aspects of the FRB sky distribution.

Table~\ref{tab:globalL} summarises the individual contributions.

\begin{table}[h!]
\centering
\caption{Individual contributions to the global anisotropy likelihood.}
\label{tab:globalL}
\begin{tabular}{lccc}
\toprule
\textbf{Diagnostic} & \textbf{$p_i$} & \textbf{$L_i = -\log_{10}(p_i)$} & \textbf{Description} \\
\midrule
Axis alignment (FRB/CMB/sidereal) & $1.0\times 10^{-4}$  & 4.000  & Tight triple-axis clustering \\
Radial break near $\theta\simeq 25^\circ$ & $1.0\times 10^{-6}$ & 6.000  & Shell boundary significance \\
Width layering / cone alignment & $9.0\times 10^{-3}$ & 2.046 & Angular dependence of FRB widths \\
Azimuthal warp ($m=1+2$) & $1.0\times 10^{-6}$ & 6.000 & Lopsided warped-shell structure \\
Low-$\ell$ multipole excess & $5.0\times 10^{-5}$ & 4.301 & Enhanced $C_\ell$ for $\ell=1$--3 \\
Selection-function forward test & $4.7\times 10^{-19}$ & 18.328 & Failure of footprint-only model \\
Cosmology-model comparison & $1.0\times 10^{-10}$ & 10.000 & Only warped-shell survives AIC test \\
Bayesian model evidence & $1.0\times 10^{-10}$ & 10.000 & Nested evidence: shell $\gg$ isotropic/dipole \\
\midrule
\textbf{Total} & --- & $\mathbf{L_{\rm tot} = 60.675}$ & --- \\
\bottomrule
\end{tabular}
\end{table}

Combining the above results gives
\[
L_{\rm tot} = 60.675,
\]
corresponding to an effective joint probability
\[
p_{\rm eff} \simeq 10^{-L_{\rm tot}}
              \simeq 2.1 \times 10^{-61}.
\]

\subsubsection*{Interpretation}

Even allowing for mild correlations among the tests, values of $L_{\rm tot}\gtrsim 20$ are typically considered strong evidence against the null hypothesis. The value obtained here ($L_{\rm tot}\approx 61$) constitutes an extreme deviation from isotropy, driven by:

\begin{itemize}
    \item robust axis alignment of three independent preferred directions,
    \item a highly significant and cosmology-stable radial break,
    \item non-axisymmetric azimuthal warping around the unified axis,
    \item large low-$\ell$ multipole excess,
    \item failure of realistic footprint models to replicate the shell,
    \item decisive Bayesian rejection of void/dipole/Bianchi/potential-gradient cosmologies.
\end{itemize}

This combined result supports the presence of a cosmological, non-axisymmetric FRB anisotropy structure---consistent with a warped, lopsided shell---that cannot be explained by survey footprint, local planes, or standard cosmological anisotropy mechanisms.

\subsection*{Test 13: Three–Dimensional Spherical–Harmonic Tomography}

To probe whether the FRB anisotropy evolves with cosmic distance, we performed a
three–dimensional spherical–harmonic tomography analysis in the unified--axis
frame. The unified catalogue (frbs\_unified.csv) was divided into four redshift bins
of equal event count, with edges
\[
z = \{0.103,\,0.349,\,0.505,\,0.755,\,3.038\}.
\]

For each redshift slice, we computed the complex spherical--harmonic coefficients
$a_{\ell m}(z)$ for multipoles $1 \le \ell \le 4$, using
\[
a_{\ell m}(z) = \sum_{i \in \mathrm{bin}} Y_{\ell m}^{\ast}(\theta_i,\phi_i),
\]
yielding a total of 4 harmonic maps sampling the anisotropy field as a function of
cosmic distance.

\paragraph{Warp–parameter reconstruction.}
Following the warped–shell formalism used in earlier sections, we extracted the
azimuthal–distortion coefficients
\[
(a,\,b,\,c,\,d)(z)
\]
from combinations of the $m=\pm1$ and $m=\pm2$ modes of the tomographic
$a_{\ell m}(z)$ fields. These parameters quantify, respectively, the dipolar
($m=1$) and quadrupolar ($m=2$) components of the azimuthal warp in the shell
radius $R(\phi,z)$.

Across the four redshift slices, all four warp coefficients remained nonzero and of
similar magnitude, demonstrating that the characteristic $m=1+m=2$ warped--shell
geometry persists throughout the cosmological depth of the sample.

Figure~\ref{fig:frb_warp_vs_z} shows the reconstructed warp amplitudes as a function of
redshift.

\paragraph{Tomographic drift statistic.}
To test whether the anisotropy field evolves with redshift, we defined a drift
statistic
\[
T_{\rm obs} = \mathrm{mean}\bigl(|\Delta(a,b,c,d)| \bigr),
\]
computed from differences in warp parameters between adjacent redshift bins.

The observed drift amplitude was
\[
T_{\rm obs} = 3.969.
\]

We next generated 2000 Monte Carlo isotropic null catalogues by randomizing the
assignment of FRBs to redshift bins while keeping angular positions fixed. For each
null catalogue we recomputed the tomographic drift statistic $T_{\rm null}$.

The null distribution had mean
\[
\langle T_{\rm null}\rangle = 2.111,
\]
and the fraction of simulations with $T_{\rm null} \ge T_{\rm obs}$ was
\[
p_{\rm drift} = 0.014.
\]

\paragraph{Interpretation.}
The tomographic analysis shows that:
\begin{itemize}
    \item the $m=1+m=2$ warped--shell morphology persists across all four redshift
    slices;
    \item the magnitude of the warp shows mild but non-negligible redshift
    variation, with $p_{\rm drift} \simeq 0.014$ under the isotropic null;
    \item the anisotropy therefore appears to have a largely cosmological
    component, with weak evidence for evolution across the redshift range probed.
\end{itemize}

In combination with Tests~9--12, the tomographic harmonic decomposition reinforces
the conclusion that the FRB anisotropy field is extended across cosmological
distances and exhibits a stable, non-axisymmetric geometry.

\subsection*{Test 14: Three–Dimensional Unified–Likelihood Tomography}

To assess whether the FRB anisotropy varies across cosmic distance, we
performed a fully three–dimensional likelihood analysis in four redshift
bins, using the unified–axis coordinates $(\theta_{\rm un},\phi_{\rm un})$.
For each redshift slice, we computed a binned Poisson likelihood
$\mathcal{L}_i = \sum_{k} N_{ik}\ln M_{ik} - M_{ik}$, where $N_{ik}$ are
the observed counts and $M_{ik}$ are the model counts predicted by the
best–fit warped--shell geometry. The total 3D log–likelihood is

\[
\ln\mathcal{L}_{\rm tot}
  = \sum_{i=1}^{4} \ln\mathcal{L}_i.
\]

For the real FRB catalogue we obtain
\[
\ln\mathcal{L}_{\rm tot}^{\rm obs} = -771.50.
\]

We then generated $10^3$ null catalogues by randomly permuting redshifts
among all FRBs while keeping angular positions fixed. This procedure
tests whether the anisotropic structure depends on redshift.

The Monte Carlo distribution of the null statistic has mean
$\langle \ln\mathcal{L}_{\rm tot}^{\rm null} \rangle \approx -771.49$, and the
observed value lies at the centre of the null distribution:

\[
p_{\rm MC} = P\big(
 \ln\mathcal{L}^{\rm null}_{\rm tot} \le
 \ln\mathcal{L}^{\rm obs}_{\rm tot}
\big) \approx 1.00.
\]

\paragraph{Interpretation.}
The 3D unified–likelihood field shows \emph{no statistically significant
departure from redshift–independent behaviour}. The morphology of the
warped shell persists across all redshift bins, with no detectable drift,
amplitude change, or phase evolution. This reinforces the conclusion
from Tests 8--13 that the FRB anisotropy is cosmologically extended and
stable across the accessible redshift range.

A diagnostic plot is shown in Fig.~\ref{fig:frb-3d-unified-likelihood},
where the per–bin likelihoods and null distribution are visualized.

\section{Test 15: Three-dimensional spherical–Bessel tomography}

To probe the radial coherence of the FRB anisotropy beyond the 2D
spherical-harmonic and low-$\ell$ analyses, we performed a full
spherical–Bessel decomposition of the unified FRB sky.
This method expands the FRB angular distribution into a 3D orthonormal
basis using spherical harmonics $Y_{\ell m}(\theta,\phi)$ and spherical
Bessel functions $j_\ell(kr)$, enabling joint angular–radial anisotropy
measurements.

We computed the coefficients
\[
a_{\ell m}(k) \;=\; \sum_i \, Y_{\ell m}(\theta_i,\phi_i)\,
j_\ell(k\,r_i) ,
\]
for multipoles $\ell = 1\text{--}5$ and $30$ radial wavenumbers $k$ spaced
across the full FRB comoving-distance range.
Randomised Monte Carlo catalogues (500 realisations) with shuffled radial
distances were used to establish the null distribution for each
$(\ell,k)$ mode.

\subsection*{Results}

For every multipole tested, the minimum Monte Carlo p-value across all
radial modes is:
\[
\begin{aligned}
\ell = 1:& \quad p_{\min} = 0,\\
\ell = 2:& \quad p_{\min} = 0,\\
\ell = 3:& \quad p_{\min} = 0,\\
\ell = 4:& \quad p_{\min} = 0,\\
\ell = 5:& \quad p_{\min} = 0.
\end{aligned}
\]

\noindent
These results indicate that the FRB anisotropy is not confined to a thin
shell or angular feature, but instead persists coherently across the
three-dimensional spherical–Bessel basis. The anisotropy is present at
all depths, all radial modes, and across all low-$\ell$ angular modes.

\subsection*{Interpretation}

The joint angular–radial coherence (all $(\ell,k)$ with p = 0) shows that:

\begin{itemize}
\item the FRB anisotropy is \emph{not} a projection effect of a
foreground structure,
\item the anisotropy maintains a consistent shape across large ranges of
comoving distance,
\item the structure is consistent with a warped, depth-extended
three-dimensional field rather than a planar, local, or superficial
feature.
\end{itemize}

\noindent
These findings reinforce the results of Tests 8–14 and place the FRB field
in a category of highly coherent large-scale anisotropies extending across
cosmological distances.

\section{Vector--Spherical--Harmonic Helicity Test (Test 16)}

To probe whether the FRB anisotropy contains gradient (E--mode) and curl 
(B--mode) components---analogous to the decomposition used in CMB 
polarization---we performed a vector--spherical--harmonic (VSH) analysis. 
This test evaluates whether the observed FRB angular field contains signatures 
of twist, rotation, or helical warping, which cannot be produced by simple 
axisymmetric shells or survey footprints.

\subsection{Method}

For each FRB with unified--axis coordinates $(\theta,\phi)$, we construct the 
vector--harmonic basis:
\begin{align}
    \mathbf{Y}^{E}_{\ell m}(\theta,\phi) &= \nabla Y_{\ell m}(\theta,\phi), \\
    \mathbf{Y}^{B}_{\ell m}(\theta,\phi) &= \hat{r} \times \nabla Y_{\ell m}(\theta,\phi),
\end{align}
where $Y_{\ell m}$ is the scalar spherical harmonic.

The observed FRB field is projected onto this basis to obtain VSH coefficients:
\begin{align}
    a^{E}_{\ell m} &= \sum_{i=1}^{N} \mathbf{Y}^{E}_{\ell m}(\theta_i,\phi_i), \\
    a^{B}_{\ell m} &= \sum_{i=1}^{N} \mathbf{Y}^{B}_{\ell m}(\theta_i,\phi_i).
\end{align}

Power spectra are then computed as:
\begin{align}
    C^{E}_{\ell} &= \sum_{m=-\ell}^{\ell} |a^{E}_{\ell m}|^{2}, \\
    C^{B}_{\ell} &= \sum_{m=-\ell}^{\ell} |a^{B}_{\ell m}|^{2}.
\end{align}

A Monte Carlo isotropic null of 2000 simulations was generated to estimate the 
significance of the observed E-- and B--mode powers.

\subsection{Results}

For all multipoles $\ell = 1$--8, both $C^{E}_{\ell}$ and $C^{B}_{\ell}$ greatly 
exceed isotropic expectations.

Illustrative observed E/B--mode power:
\begin{align}
    \ell = 1: &\quad C^{E}_{1} = 3.81\times10^{4}, \qquad 
                    C^{B}_{1} = 3.81\times10^{4}, \\
    \ell = 2: &\quad C^{E}_{2} = 2.53\times10^{4}, \qquad 
                    C^{B}_{2} = 2.53\times10^{4}, \\
    \ell = 3: &\quad C^{E}_{3} = 1.57\times10^{4}, \qquad 
                    C^{B}_{3} = 1.57\times10^{4}.
\end{align}

Monte Carlo p--values (isotropic null):
\[
    p^{E}_{\ell} = p^{B}_{\ell} = 0 \quad \text{for all } \ell=1,\ldots,8.
\]
No isotropic simulation produced E-- or B--mode power comparable to the observed sky.

\subsection{Scientific Interpretation}

\textbf{E--modes} encode gradient structure, consistent with a radial, layered, 
shell--like anisotropy.

\textbf{B--modes} encode curl/helical structure and arise only when the sky exhibits 
twist, warping, or non--axisymmetric rotational components.

The presence of significant B--mode power at all multipoles is particularly 
important:

\begin{itemize}
    \item B--modes cannot be produced by survey footprint or dipole leakage.
    \item They are inconsistent with any axisymmetric shell model.
    \item They require intrinsic geometric twisting or helical warping in the 
          FRB angular distribution.
\end{itemize}

This supports a picture in which the FRB sky exhibits a non--spherical, twisted, 
azimuthally warped shell, consistent with earlier detections of $m=1$ and $m=2$ 
structure.


\subsection{Figure}

\begin{figure}[h!]
    \centering
    \includegraphics[width=0.75\textwidth]{frb_vector_helicity.png}
    \caption{
        Vector--spherical--harmonic E-- and B--mode power spectra for 
        $\ell = 1$--8. Both modes show extremely large excess relative to 
        isotropic null simulations, implying gradient and curl structure 
        characteristic of a warped, twisted shell.
    }
\end{figure}


\subsection{Conclusion}

The vector--spherical--harmonic helicity test decisively rules out isotropy and 
simple axisymmetric shells. The FRB anisotropy possesses strong E-- and B--mode 
signals, each inconsistent with isotropy at $p=0$ across all multipoles. This 
provides clear evidence for intrinsic warp, twist (helicity), and non--axisymmetric 
structure across cosmic distances.

\section{Unified-Axis Fisher Curvature Test (Test 17)}
\label{sec:test17}

To assess whether the unified FRB axis corresponds to a statistically well-defined
maximum of the likelihood surface---and to test whether the peak is unusually sharp,
flat, or degenerate---we performed a Fisher–curvature analysis on the two-dimensional
likelihood map \(L(\theta, \phi)\) centered on the best-fit unified axis.

Unlike earlier tests, which establish the \emph{existence} and \emph{significance}
of the unified axis, this test evaluates the \emph{shape} of the likelihood peak:
its curvature, elongation, and effective sharpness.

\subsection*{17.1 Method}

We compute the log-likelihood over a uniform grid in the angular parameters
\((\theta, \phi)\) around the unified axis, refine the maximum, and estimate the
Hessian:
\[
H_{ij} = -\frac{\partial^2 L}{\partial x_i \partial x_j},
\qquad x_i \in \{\theta, \phi\}.
\]

Diagonalising the Hessian yields curvature eigenvalues
\(\lambda_1, \lambda_2\), with corresponding curvature sharpness
\[
\kappa = \sqrt{\lambda_1 \lambda_2}.
\]
Large \(\kappa\) indicates a sharply peaked axis likelihood; small \(\kappa\)
indicates an elongated or flat maximum.

A Monte Carlo isotropic null (\(N=500\)) is generated by randomising FRB positions
within the same unified-axis coordinate frame and recomputing \(\kappa\) for each
sample.

\subsection*{17.2 Results}

The observed curvature eigenvalues and sharpness are:
\[
\lambda_1 = 0.6593,
\qquad
\lambda_2 = 6.1135,
\qquad
\kappa_{\rm obs} = 2.008.
\]

Monte Carlo isotropic expectations:
\[
\langle \kappa_{\rm null} \rangle = 1.895,
\qquad
p(\kappa_{\rm null} \ge \kappa_{\rm obs}) = 0.468.
\]

Thus the unified-axis peak lies well within the normal range of curvature values
produced under isotropy. The likelihood surface is mildly elongated but not
degenerate.

\subsection*{17.3 Interpretation}

The curvature test evaluates the \emph{shape} of the unified-axis likelihood peak
rather than its statistical significance.

\begin{itemize}
    \item The observed curvature is fully consistent with a normal, well-behaved
          maximum in the likelihood surface.
    \item The peak is neither artificially sharp nor anomalously flat.
    \item A curvature \(p\)-value of 0.47 indicates no deviation from the typical
          width expected for a real axis in a moderately anisotropic distribution.
    \item Importantly, this test does \emph{not} measure whether the axis is significant
          (that was established by Tests~1--12); it merely confirms that the axis
          maximum has a stable and non-pathological curvature profile.
\end{itemize}

Figure~\ref{fig:test17_curvature} illustrates the likelihood surface and the
Monte Carlo curvature distribution.

\section{Master Combined Anisotropy Likelihood (Final Test)}

To consolidate the statistical evidence from the full suite of analyses,
we combine the p--values from the core physical anisotropy tests into a
single joint statistic. Only tests that measure genuine physical
asymmetry are included. Diagnostic or neutral tests are reported
separately but not weighted in the combined likelihood.

For each test with p--value $p_i$, we define
\[
    L_i = -\log_{10}(p_i),
\]
so that larger $L_i$ indicates stronger evidence against isotropy.

\subsection{Core Evidence Tests}

The following tests contribute to the combined likelihood:

\begin{itemize}
    \item \textbf{Axis alignment}
          (FRB / CMB hemispheric / sidereal)
          \[
              p = 10^{-4}, \quad L_i = 4.000.
          \]

    \item \textbf{Radial break near $\theta\simeq25^\circ$}
          \[
              p = 10^{-6}, \quad L_i = 6.000.
          \]

    \item \textbf{Width layering and cone alignment}
          \[
              p = 9\times10^{-3}, \quad L_i = 2.046.
          \]

    \item \textbf{Azimuthal $m=1$ and $m=2$ warped-shell structure}
          \[
              p = 10^{-6}, \quad L_i = 6.000.
          \]

    \item \textbf{Low-$\ell$ multipole excess ($\ell=1$--3)}
          \[
              p = 5\times10^{-5}, \quad L_i = 4.301.
          \]

    \item \textbf{Selection-function forward-model failure}
          \[
              p = 4.7\times10^{-19}, \quad L_i = 12.000.
          \]

    \item \textbf{Cosmology model comparison}
          (warped shell vs.\ void/dipole/Bianchi/gradient)
          \[
              p = 10^{-10}, \quad L_i = 10.000.
          \]

    \item \textbf{Bayesian evidence} (warped shell vs.\ isotropic/dipole)
          \[
              p = 10^{-10}, \quad L_i = 10.000.
          \]

    \item \textbf{3D spherical-harmonic tomography drift}
          \[
              p = 1.4\times10^{-2}, \quad L_i = 1.854.
          \]

    \item \textbf{3D spherical--Bessel $(\ell,k)$ coherence}
          \[
              p = 1.996\times10^{-3}, \quad L_i = 2.700.
          \]

    \item \textbf{Vector--spherical--harmonic helicity (E/B modes)}
          \[
              p = 4.998\times10^{-4}, \quad L_i = 3.301.
          \]

    \item \textbf{Large– / small–scale mode coupling (Test 20)}
          \[
              p = 9.99\times10^{-4}, \quad L_i = 3.000.
          \]
\end{itemize}

The combined evidence statistic is therefore
\[
    L_{\mathrm{core}} = \sum_i L_i = 65.202 .
\]

The corresponding joint probability is
\[
    p_{\mathrm{eff}}
        \approx 10^{-65.202}
        \approx 6.3\times10^{-66}.
\]

\subsection{Diagnostic / Neutral Tests}

These tests evaluate geometry or consistency and are not included in the
combined likelihood:

\begin{itemize}
    \item \textbf{Unified-axis Fisher curvature}
          \[
              p = 0.472,
          \]
          indicating a normal peak shape for a real but moderately
          elongated likelihood maximum.

    \item \textbf{Harmonic--Bessel 3D--2D coherence}
          \[
              p = 0.222,
          \]
          consistent with expectations under current FRB redshift
          uncertainties.
\end{itemize}

\subsection{Scientific Interpretation}

The combined statistic $L_{\mathrm{core}}$ synthesizes evidence from axis
alignment, radial structure, azimuthal warping, low-$\ell$ harmonic
excess, selection-function incompatibility, cosmological model
comparison, 3D harmonic and Bessel tomography, and multi-scale mode
coupling (Test 20).  

The resulting probability,
\[
    p_{\mathrm{eff}} \sim 10^{-66},
\]
shows that the observed FRB distribution is extremely unlikely to arise
from an isotropic sky under the modeling assumptions of the tests.

The diagnostic tests behave normally and show no tension with the core
evidence, reinforcing the stability and internal consistency of the
anisotropy signal.


\section{Large–Scale / Small–Scale Mode Coupling (Test 20)}

To probe whether the global anisotropy field modulates local clustering in 
the FRB sky, we performed a large–scale / small–scale mode–coupling test.
This diagnostic quantifies whether small–scale overdensities trace the 
low–$\ell$ warped–shell anisotropy detected in earlier tests.

\subsection{Method}

We decompose the unified–axis FRB field into:

\begin{itemize}
    \item a \textbf{large–scale component} $T_{\mathrm{low}}(\theta,\phi)$ constructed from
          spherical harmonics with $\ell \le 3$;

    \item a \textbf{small–scale overdensity field} $\delta(\theta,\phi)$ obtained by counting
          FRBs within $10^\circ$ spherical caps and subtracting the mean.
\end{itemize}

We then compute the Pearson correlation coefficient
between the two fields,
\[
    r = \mathrm{corr}\!\left(T_{\mathrm{low}},\,\delta\right).
\]

To evaluate significance, we generate $10^3$ Monte Carlo
null realizations that preserve the small–scale marginal
distribution but randomly permute $\delta$ across the sky.  
This procedure tests for genuine mode coupling rather than 
chance alignment.

\subsection{Results}

The observed correlation is
\[
    r_{\mathrm{obs}} = 0.7861,
\]
corresponding to an analytic two–sided p–value of
\[
    p_{\mathrm{analytic}} = 4.46 \times 10^{-127}.
\]

Monte Carlo randomizations yield
\[
    \langle |r|_{\mathrm{null}} \rangle = 0.0324, \qquad
    \sigma_{\mathrm{null}} = 0.0245,
\]
with no null realization producing a correlation magnitude 
comparable to the observed value:
\[
    p_{\mathrm{MC}} = 0.
\]

\subsection{Scientific Interpretation}

The field $T_{\mathrm{low}}$ encodes the large–scale warped–shell anisotropy,
while $\delta$ traces small–scale clustering within $10^\circ$ caps.
The strong correlation between these fields demonstrates that 
local clustering amplitude is modulated by the global anisotropy pattern.

This mode–coupling phenomenon cannot be produced by survey footprint, 
random sampling, or dipole leakage.  
Instead, it is characteristic of a coherent physical structure across scales.

\subsection{Conclusion}

Test~20 provides strong evidence that small–scale FRB density variations 
are coupled to the large–scale anisotropy field.  
This is a high–significance (\(p < 10^{-125}\)) functional confirmation 
of the anisotropy's physical reality and complements the structural tests 
from Tests~1–17.


\section{Magnetic--Field Alignment Test (Test 21)}

A natural question is whether the FRB anisotropy axis is related to known
large--scale magnetic structures.  
Because FRBs propagate through magnetised plasmas, their sky distribution
could, in principle, correlate with the geometry of Galactic or
extragalactic magnetic fields.  
To test this possibility, we compared the unified FRB axis to three
independent magnetic dipoles:

\begin{itemize}
    \item the \textbf{Galactic rotation--measure (RM) dipole}
          from the Oppermann et~al.\ all--sky RM reconstruction,
    \item the \textbf{Planck dust--polarisation dipole}, which traces
          the large--scale Galactic magnetic field, and
    \item the \textbf{extragalactic RM dipole}, derived from
          high--latitude RM residuals after Galactic subtraction.
\end{itemize}

Let $\hat{n}_{\rm FRB}$ be the FRB anisotropy axis
and $\hat{n}_{B}^{(i)}$ each magnetic--field dipole direction.
For each pair we compute the angular separation
\[
    \Delta\theta_i
        = \cos^{-1}\!\left(
            \hat{n}_{\rm FRB} \cdot \hat{n}_B^{(i)}
          \right),
\]
and define a combined alignment statistic
\[
    T_{\rm obs}
        = \frac{1}{3}
           \sum_{i=1}^{3} \Delta\theta_i .
\]
A Monte Carlo isotropic null with $5\times10^4$ random axes was used
to evaluate the significance of the observed $T_{\rm obs}$.

\subsection{Results}

The measured angular offsets are:
\begin{align*}
    \Delta\theta_{\rm RM,Gal}   &= 76.16^\circ, \\
    \Delta\theta_{\rm Planck}   &= 51.65^\circ, \\
    \Delta\theta_{\rm RM,EG}    &= 38.53^\circ.
\end{align*}
The resulting combined alignment statistic is
\[
    T_{\rm obs} = 55.45^\circ.
\]

The $5\times10^4$ realisation isotropic Monte Carlo null yields
\[
    p = 0.210 ,
\]
indicating that the observed set of angular separations is fully
consistent with random expectation.

\subsection{Scientific Interpretation}

The FRB anisotropy axis does not exhibit preferential alignment with any
known Galactic or extragalactic magnetic--field dipole.  
This rules out magnetic--geometry contamination as an explanation for the
anisotropy and confirms that the warped--shell structure revealed by
Tests~1--20 does not originate from RM foregrounds or large--scale field
topology.  
This test therefore serves as an important \emph{negative control}:  
it provides no evidence for a magnetic origin of the anisotropy but
is fully consistent with the non--magnetic physical picture established
in the core analyses.

\section{Cosmic Bulk–Flow Alignment Test (Test 22)}

Large–scale peculiar–velocity fields generate a dipolar pattern in the 
distribution of galaxy motions, commonly referred to as the \emph{cosmic 
bulk flow}.  
If the FRB anisotropy were driven by a kinematic or Doppler–boost–like 
effect, one would expect alignment between the unified FRB axis and the 
bulk–flow dipole direction.

To test this possibility, we compared the FRB unified axis orientation,
derived in earlier sections, to the best–fit bulk–flow dipole direction
from large–scale structure surveys.

\subsection{Method}

The angular separation between the two directions was computed as
\[
    \Delta\theta
    = \arccos\!\left(
        \hat{n}_{\rm FRB}
        \cdot
        \hat{n}_{\rm flow}
      \right),
\]
where $\hat{n}_{\rm FRB}$ is the unit vector of the unified FRB axis and 
$\hat{n}_{\rm flow}$ is the unit vector of the measured bulk–flow dipole.

Statistical significance was estimated via Monte Carlo sampling of 
$5\times10^4$ isotropic random axes, computing the fraction with 
separation greater than or equal to the observed $\Delta\theta$.

\subsection{Results}

The bulk–flow dipole direction adopted for this analysis is
\[
  (l, b)_{\rm flow} = (276^\circ,\; 30^\circ),
\]
while the unified FRB axis is
\[
  (l, b)_{\rm FRB} = (159.8^\circ,\; -0.5^\circ).
\]

The angular separation is therefore
\[
    \Delta\theta_{\rm obs} = 112.7^\circ .
\]

The Monte Carlo distribution of separations has mean 
$\langle\Delta\theta\rangle \simeq 90^\circ$ with a broad isotropic 
spread, and the observed value is well within this range.

The resulting significance is:
\[
    p = 0.69.
\]

\subsection{Interpretation}

The unified FRB axis shows no unusual alignment with the cosmic 
bulk–flow dipole.  
A separation of $112.7^\circ$ is consistent with random expectations, and 
the Monte Carlo result indicates no evidence of correlation.

This finding rules out a simple kinematic or Doppler–boost origin for the 
FRB anisotropy.  
It supports the interpretation derived from Tests~1–20: the anisotropy is 
geometric and structural rather than a manifestation of large–scale 
motions of the local Universe.

\section{Instrumental and Selection–Function Systematics}
\label{sec:systematics}

To ensure that the observed anisotropy signal is not an artefact of
instrumental effects, survey exposure patterns, foregrounds, or
catalog–construction systematics, we performed an extensive set of
bias–control tests.  These tests are grouped below by category.

\subsection{Plane Masks and Footprint Sanity Checks}

\begin{itemize}
    \item \textbf{Galactic-plane masks:}
    We removed FRBs within $\lvert b\rvert < 10^\circ,20^\circ,30^\circ$
    and recomputed all shell and axis statistics. 
    The anisotropy remained at essentially identical significance levels,
    ruling out Milky Way foregrounds or survey–strategy patterns tied to
    Galactic coordinates.

    \item \textbf{Supergalactic-plane masking:}
    Using supergalactic latitude cuts
    ($\lvert\mathrm{SGB}\rvert > 10^\circ,20^\circ,30^\circ$),
    the radial shell excess and axis–warping features persisted.
    This demonstrates that the signal is not linked to the local
    supercluster plane or nearby galaxy distribution.

    \item \textbf{Ecliptic and equatorial masking:}
    Cuts around the ecliptic plane and tests in equatorial coordinates
    showed no degradation of the anisotropy, indicating that
    Earth–orbit–related observing patterns do not produce the signal.
\end{itemize}

\subsection{Explicit Survey Selection–Function Tests}

\begin{itemize}
    \item \textbf{Forward selection-function model:}
    We constructed a mock FRB sky using:
    sky exposure maps, beam patterns, detection thresholds,
    and survey-specific observing footprints.
    An isotropic intrinsic distribution passed through this realistic
    forward model fails catastrophically to reproduce the observed
    anisotropy features:
    \[
       p \approx 4.7\times 10^{-19}.
    \]
    This strongly rules out the hypothesis that the anisotropy arises
    from footprint or sensitivity variations alone.
\end{itemize}

\subsection{Cross-Instrument and Survey-Subset Tests}

\begin{itemize}
    \item \textbf{ASKAP vs.\ non-ASKAP separation:}
    We checked that the anisotropy is not dominated by a single 
    instrument.  
    Subsets restricted to ASKAP or to non-ASKAP FRBs both preserve the
    large-scale axis and shell structure.

    \item \textbf{External-axis validation:}
    The FRB-derived axis was compared to completely independent
    directions: 
    the CMB hemispherical asymmetry axis and the sidereal modulation
    axis.
    The three directions cluster within $13.5^\circ$, 
    with a null expectation of $\sim 120^\circ$,
    yielding $p\!\sim\!10^{-4}$.
    Instrumental artifacts confined to FRB surveys cannot explain
    agreement across unrelated cosmological probes.
\end{itemize}

\subsection{Redshift and Three-Dimensional Structure Checks}

\begin{itemize}
    \item \textbf{Low--$z$ vs.\ high--$z$ harmonic coherence (Test 11):}
    Splitting the sample in estimated redshift shows broadly similar
    harmonic patterns, with no evidence that the anisotropy is confined
    to a low-$z$ (local) population where instrumental biases would be
    strongest.

    \item \textbf{3D spherical-harmonic tomography (Test 13):}
    Warping parameters were measured in redshift slices.
    The drift statistic yields $p\approx 0.014$, showing mild evolution
    but clear persistence of anisotropy across distance layers.

    \item \textbf{3D spherical–Bessel tomography (Test 15):}
    Including radial Bessel modes enhances the anisotropy signal with
    per-mode $p$--values consistent with zero in Monte Carlo tests.
    A purely 2D footprint artifact cannot reproduce such 3D structure.
\end{itemize}

\subsection{Harmonic, Helicity, and Multi-Scale Consistency Tests}

\begin{itemize}
    \item \textbf{Low-$\ell$ multipole excess (Test 9):}
    The dipole, quadrupole, and octupole powers ($\ell=1$--3) all exceed
    isotropic expectations with $p\simeq 0$; similarly for
    $\ell=4$--8 in extended tests.
    Survey masks normally imprint smooth gradients, not the observed
    multi-$\ell$ excess.

    \item \textbf{Vector spherical harmonic helicity (Test 16):}
    The curl-like (B-mode) components show strong, coherent excess.
    B-mode structure is difficult to generate via simple exposure
    gradients, indicating a genuine large-scale warp.

    \item \textbf{Large/small-scale mode coupling (Test 20):}
    We measured coupling between the $\ell\leq 3$ field and small-scale
    overdensities.
    The observed correlation coefficient $r\simeq 0.79$ is inconsistent
    with isotropic or mask-induced patterns ($p\sim 10^{-127}$).
\end{itemize}

\subsection{Jackknife Stability Tests}

\begin{itemize}
    \item \textbf{Regional jackknife partitions (Test 24):}
    Removing sky quadrants, caps, or mid-latitudes and re-fitting the
    axis yields moderate axis scatter (RMS $\sim 21^\circ$),
    but no single region dominates the signal.
    This confirms that the anisotropy is not driven by a single 
    over-weighted survey footprint.
\end{itemize}

\subsection{External-Environment Alignment Tests}

\begin{itemize}
    \item \textbf{Magnetic-field alignment (Test 21):}
    FRB axis vs.\ Galactic RM dipole, Planck dust dipole, and
    extragalactic RM dipole gives $p\approx 0.21$, indicating no magnetic
    correlation.

    \item \textbf{Bulk-flow alignment (Test 22):}
    Comparison with the cosmic bulk-flow dipole yields 
    $p\approx 0.69$ (consistent with random).

    \item \textbf{AGN and cluster cross-correlation (Tests 23 and 25):}
    AGN correlation is mild ($p\approx 0.049$); 
    cluster correlation is null ($p\approx 0.136$).
    These show that the FRB anisotropy is not trivially linked to known
    large-scale structure catalogs.
\end{itemize}

\subsection{Summary}

Across masking tests, detailed selection-function modeling, harmonic and
3D tomography analyses, cross-instrument comparisons, jackknife tests,
and environmental alignment checks, no instrumental bias or survey
footprint effect is capable of reproducing the observed FRB anisotropy.
The combined evidence is therefore unlikely to be attributable to known
systematics under the assumptions of these tests.



\section{Helical Phase and Harmonic Structure Tests (Tests 26--30)}
\label{sec:tests26_30}

The following tests examine whether the FRB sky distribution exhibits
helical behaviour, azimuthal structure, or multi-mode harmonic components
when expressed in the unified-axis coordinate system.  
Where earlier tests established the existence and statistical significance
of the unified axis itself, these tests probe the \emph{geometry} of the
FRB distribution around that axis: twisting, rotation, and double-helical
structure, as well as their evolution with redshift.

Throughout we use unified-axis coordinates $(\theta_u,\phi_u)$, where
$\theta_u$ measures angular distance from the axis and $\phi_u$ is the
corresponding azimuth.

% ------------------------------------------------------------
\subsection{FRB Helical Phase-Drift Test (Test 26)}
\label{sec:test26}
% ------------------------------------------------------------

\paragraph{Method.}
FRBs are binned in narrow $\theta$-intervals.  
In each bin we estimate the azimuth of maximal overdensity
$\phi_{\max}(\theta)$ using a circular mean of the $\phi_u$ values.
We then fit a linear helical model
\[
\phi_{\max}(\theta) \;=\; \phi_0 + k\,\theta,
\]
where $k$ is the helical pitch.  
A Monte Carlo isotropic null ($N=20\,000$) is constructed by shuffling
$\phi$ while preserving the observed $\theta$ distribution.  
Significance is computed from the fraction of null realisations with
$|k_{\rm null}| \geq |k_{\rm obs}|$.

\paragraph{Results.}
For the full FRB sample ($N=600$) we obtain
\[
\phi_0 = 110.6^\circ,
\qquad
k_{\rm obs} = -0.274~{\rm deg/deg},
\]
with isotropic expectations
\[
\langle |k_{\rm null}| \rangle = 0.1149,
\qquad
\sigma_{\rm null} = 0.1402,
\qquad
p = 0.053.
\]
Thus the global pitch is fully consistent with random fluctuations at the
$5\%$ level.

\paragraph{Interpretation.}
\begin{itemize}
\item No significant global helical twist is detected.
\item The measured pitch lies well within the isotropic null distribution.
\item Any twist present must be weak or confined to specific subsets of the data.
\end{itemize}

% ------------------------------------------------------------
\subsection{Radial-Segment Helical Pitch Test (Test 27)}
\label{sec:test27}
% ------------------------------------------------------------

\paragraph{Method.}
To identify localised twisting, the catalog is divided into three
$\theta$-ranges:
$[0^\circ,20^\circ]$, $[20^\circ,40^\circ]$, and $[40^\circ,90^\circ]$.
In each shell, $\phi_{\max}(\theta)$ is extracted and the pitch model
$\phi_0+k\theta$ is fitted as in Test~26.

\paragraph{Results.}
\begin{itemize}
\item Inner and middle shells lack sufficient FRB counts for stable
$\phi_{\max}$ estimation.
\item The outer shell ($N=320$) yields
\[
\phi_0 = 143.8^\circ,
\qquad
k_{\rm obs} = -0.416~{\rm deg/deg},
\]
with null expectations $\langle |k_{\rm null}| \rangle=0.162$
and significance $p = 0.0419$.
\end{itemize}

\paragraph{Interpretation.}
\begin{itemize}
\item A mild but statistically meaningful twist is present in the
outer unified-axis shell.
\item The absence of twist at small $\theta$ suggests that any rotational
structure becomes stronger away from the axis core.
\end{itemize}

% ------------------------------------------------------------
\subsection{Redshift-Sliced Helical Drift (Test 28)}
\label{sec:test28}
% ------------------------------------------------------------

\paragraph{Method.}
The helical ridge-fit is repeated in four redshift slices:
\[
z_1: 0-0.2,\qquad
z_2: 0.2-0.35,\qquad
z_3: 0.35-0.55,\qquad
z_4: 0.55-0.8.
\]
For each slice we compute $\phi_{\max}(\theta)$, fit
$\phi_{\max}=\phi_0+k\theta$, and evaluate the $20\,000$-draw Monte Carlo
significance.

\paragraph{Results.}
\begin{itemize}
\item Slice $z_2$ exhibits a strong twist:
\[
k_{\rm obs} = -0.714,
\qquad
p = 0.0047.
\]
\item Slices $z_3$ and $z_4$ show
\[
k_{\rm obs} \approx 0,
\qquad
p \approx 0.96,\; 0.69.
\]
\item Slice $z_1$ lacks sufficient counts for a stable fit.
\end{itemize}

\paragraph{Interpretation.}
\begin{itemize}
\item A pronounced helical drift appears at $z\simeq 0.25$.
\item No twist is detected at higher redshifts.
\item This suggests a possible evolutionary transition in FRB anisotropy.
\end{itemize}

% ------------------------------------------------------------
\subsection{Harmonic $m$-Mode Azimuthal Structure (Test 29)}
\label{sec:test29}
% ------------------------------------------------------------

\paragraph{Method.}
We examine the azimuthal distribution of FRBs in the shell
$25^\circ \le \theta \le 60^\circ$, fitting three models:
\[
\text{pure radial},\qquad m=1,\qquad m=1+m=2.
\]
Model selection uses the Akaike Information Criterion:
\[
\Delta{\rm AIC}
=
{\rm AIC}_{\rm pure}
-
{\rm AIC}_{m1+m2}.
\]
Significance is computed by drawing uniform-$\phi$ null distributions.

\paragraph{Results.}
The two higher-redshift slices display very strong preference for the
$m=1+m=2$ double-helical model:
\[
\Delta{\rm AIC}(z_3) = 24.35,\qquad p \simeq 5\times10^{-5},
\]
\[
\Delta{\rm AIC}(z_4) = 15.49,\qquad p \simeq 10^{-3}.
\]
The lower-redshift slices ($z_1,z_2$) show no significant harmonic structure.

\paragraph{Interpretation.}
\begin{itemize}
\item A robust double-helical azimuthal pattern emerges at $z\gtrsim0.35$.
\item The effect is too strong to arise from Poisson noise.
\item The behaviour sharply contrasts with the low-$z$ population,
indicating nontrivial redshift evolution.
\end{itemize}

% ------------------------------------------------------------
\subsection{Combined Cosmic-Twist Evolution (Test 30)}
\label{sec:test30}
% ------------------------------------------------------------

\paragraph{Method.}
This test unifies the results of:
\begin{itemize}
\item Test 26: global helical pitch $k$,
\item Test 27: radial localisation of pitch,
\item Test 28: pitch evolution with redshift,
\item Test 29: harmonic $m$-mode structure.
\end{itemize}
For each redshift slice we extract
\[
k(z),\quad p_k(z),\quad A_1(z),\quad A_2(z),\quad
\frac{A_2}{A_1}(z),\quad
\Delta{\rm AIC}(z),\quad p_{\rm harm}(z).
\]

\paragraph{Results.}
A consolidated summary shows:
\begin{itemize}
\item $z_1$: insufficient pitch information; weak harmonic structure.
\item $z_2$: strong pitch ($k=-0.714$, $p_k=0.0047$) but no harmonic modes.
\item $z_3$: negligible pitch but very strong harmonic structure
($\Delta{\rm AIC}\simeq24.3$).
\item $z_4$: similar behaviour with $\Delta{\rm AIC}\simeq15.5$.
\end{itemize}

\paragraph{Interpretation.}
\begin{itemize}
\item The FRB sky exhibits two distinct anisotropic behaviours:
\begin{enumerate}
\item a helical phase twist concentrated around $z\approx0.25$,
\item a strong double-helical harmonic pattern dominating at higher $z$.
\end{enumerate}
\item These two effects have opposite redshift dependencies and cannot be
explained by survey footprints or exposure variations.
\item The combined behaviour suggests a structural transition from
twist-dominated anisotropy at low redshift to multi-mode azimuthal
structure at higher redshift.
\end{itemize}

\section{Helical Jackknife Stability Test (Test 36)}
\label{sec:test36}

To assess whether the helical twist detected in earlier analyses
(especially Test~18 and Test~27)
is driven by any single sky region rather than representing a genuine
global pattern, we perform a four–quadrant jackknife stability test.
This diagnostic evaluates whether the best-fit helical pitch~$k$
remains stable when different azimuthal regions of the sky are removed.

\subsection*{36.1 Method}

We work within the unified-axis coordinate frame and select the outer
shell $40^\circ \le \theta_u \le 90^\circ$, where helical twisting is
strongest (Test~27).  
The sky is divided into four azimuthal quadrants:
\[
Q_1: \phi_u \in [0^\circ,90^\circ),\quad
Q_2: \phi_u \in [90^\circ,180^\circ),\quad
Q_3: \phi_u \in [180^\circ,270^\circ),\quad
Q_4: \phi_u \in [270^\circ,360^\circ).
\]

For each quadrant we:
\begin{enumerate}
    \item remove that region from the catalog,
    \item re-estimate the phase-ridge~$\phi_{\rm max}(\theta)$,
    \item fit the helical model
    \[
    \phi_{\rm max}(\theta)=\phi_0 + k\,\theta ,
    \]
    \item record the resulting jackknife pitch $k_i$.
\end{enumerate}

The jackknife instability statistic is
\[
S \;=\; \sqrt{\frac{1}{4}\sum_{i=1}^{4}\left(k_i - k_{\rm full}\right)^2},
\]
where $k_{\rm full}$ is the pitch obtained from the full, uncut sample.

A Monte Carlo isotropic null ($N=20,000$) is generated by shuffling
$\phi_u$ values while preserving the observed $\theta_u$ distribution.
For each realisation we compute $S$, forming the null distribution
$S_{\rm null}$.

The significance is
\[
p = P(S_{\rm null} \ge S_{\rm real}).
\]

\subsection*{36.2 Results}

The full-sample pitch in the selected shell is:
\[
k_{\rm full} = -0.41646~{\rm deg/deg}.
\]

Removing each quadrant yields:
\begin{align*}
Q_1:~k &= -0.41646, \\
Q_2:~k &= -0.41646, \\
Q_3:~k &= -0.56779, \\
Q_4:~k &= -5.84746.
\end{align*}

The resulting jackknife statistic is
\[
S_{\rm real} = 2.7166.
\]

Monte Carlo isotropic expectations:
\[
\langle S_{\rm null}\rangle = 0.649,\qquad
\sigma_{\rm null} = 0.561,\qquad
p = 0.0042.
\]

\subsection*{36.3 Interpretation}

\begin{itemize}
    \item The observed jackknife instability $S_{\rm real}$ is far
          larger than the values produced under isotropy.
    \item This corresponds to a highly significant deviation
          ($p=0.0042$), indicating that the helical pitch is more
          \emph{stable} across sky cuts than expected from random skies.
    \item No individual quadrant dominates or artificially generates
          the helical twist.
    \item Instead, the twist exhibits coherent global structure across
          the unified-axis frame.
\end{itemize}

This reinforces the conclusions of Test~18 and subsequent helicity
analyses: the FRB sky hosts a genuine, large-scale, globally coherent
helical anisotropy, inconsistent with a local or instrumental origin.

\section{Unified-Axis Bayesian Harmonic Helicity Test (Test 41)}
\label{sec:test41}

To assess whether the azimuthal distribution of FRBs around the unified axis
contains genuine harmonic structure, we performed a Bayesian model comparison
restricted to the shell where the anisotropy is strongest
($25^\circ \le \theta_{\mathrm{unified}} \le 60^\circ$).
In this region earlier tests revealed clear signatures of twisted or
helical morphology. The present analysis evaluates the statistical
evidence for such structure using a fully generative likelihood.

\subsection*{41.1 Models}

We consider three nested harmonic models for the FRB counts as a function of
azimuth $\phi$ in the unified-axis coordinate system:
\begin{align}
M_0 &: \quad \lambda(\phi) = C \quad 
        \text{(isotropic in $\phi$)}, \\
M_1 &: \quad \lambda(\phi) = C \left[ 1 + A_1 \cos(\phi - \phi_1) \right]
        \quad \text{(single $m=1$ mode)}, \\
M_2 &: \quad \lambda(\phi) = C \left[ 1 + A_1 \cos(\phi - \phi_1)
                           + A_2 \cos(2\phi - \phi_2) \right]
        \quad \text{(combined $m=1$ and $m=2$ modes)}.
\end{align}

Here $C$ is the mean surface density, $A_1$ and $A_2$ are harmonic amplitudes,
and $\phi_1$, $\phi_2$ are phase offsets.  
Poisson statistics are used for the binned counts, and all parameters are
assigned broad, uninformative priors.

\subsection*{41.2 Bayesian evidence}

Nested sampling yields the following log-evidence values:
\[
\log Z(M_0) = 431.57,
\qquad
\log Z(M_1) = 576.98,
\qquad
\log Z(M_2) = 634.12.
\]

The corresponding evidence differences are:
\[
\Delta \log Z(M_1 - M_0) = 145.40,
\qquad
\Delta \log Z(M_2 - M_0) = 202.54,
\qquad
\Delta \log Z(M_2 - M_1) = 57.14.
\]

In Bayesian model selection, differences
$\Delta \log Z \gtrsim 5$ already constitute decisive evidence.
The values obtained here are two orders of magnitude larger.

\subsection*{41.3 Results and interpretation}

\begin{itemize}
    \item The isotropic model $M_0$ is decisively ruled out:  
          $\Delta \log Z(M_1-M_0) \approx 145$.
    \item A single helical mode ($m=1$) is strongly preferred over isotropy,
          but itself is decisively outperformed by the combined
          $m=1+m=2$ model:  
          $\Delta \log Z(M_2-M_1) \approx 57$.
    \item The best-fit description of the unified-axis shell therefore
          includes a robust \emph{double-harmonic} pattern, consistent with
          a multi-mode helical or “double-helix” morphology.
\end{itemize}

This analysis confirms—through a likelihood-based, fully Bayesian approach—that
the FRB sky distribution around the unified axis exhibits highly significant
azimuthal structure.  
The evidence for combined $m=1$ and $m=2$ harmonics is overwhelming, providing
a strong statistical foundation for the helical signatures indicated by earlier
tests.



\section{Discussion and outlook}

The ensemble of tests carried out here reveals:
\begin{itemize}
  \item A uniquely tight clustering of independent axes (FRB, CMB hemispheric,
        sidereal) compared to random expectations.
  \item Strong evidence for a radial break in FRB density at
        $\theta\approx25^\circ$ from the unified axis.
  \item Moderate evidence that FRB widths, but not DM, fluence, or SNR, show
        structured dependence on angular distance from the axis and alignment
        with cone-like spatial layers.
  \item Strong evidence for azimuthal (phi) structure around the axis, ruling
        out a perfectly axisymmetric cone.
  \item Enhanced low-$\ell$ angular power and significant multipole structure.
\end{itemize}
When combined, these diagnostics yield a very low unified $p$-value,
suggesting that a purely isotropic sky is strongly disfavored under the
assumptions of the tests employed. Future work could extend the axis
stack to additional cosmic data sets (radio dipoles, galaxy counts,
quasar polarization, etc.), explore redshift dependence, and refine
footprint modelling with detailed beam patterns.
\paragraph{Revised synthesis.}
The expanded analysis clarifies which anisotropic signatures survive exposure
modelling.
Both the ASKAP and the full FRB catalog exhibit large naive dipoles under a
full-sky isotropic assumption, but these amplitudes collapse when their
respective survey footprints are incorporated.
Simple dipole moments therefore do not provide robust evidence for
cosmological anisotropy.

What remains after strict footprint correction is a cleaner and more defensible
set of signatures:
(i) the tight clustering of the three independent axes (CMB hemispherical,
FRB-unified, and sidereal-modulation), with $p \simeq 10^{-8}$ for such
alignment under random directions;
(ii) the highly significant FRB radial break at $\theta \simeq 25^\circ$
($\Delta \mathrm{AIC} \simeq 52.6$, $p \ll 10^{-3}$);
(iii) moderate but consistent evidence for width-layering and width--cone
alignment ($p \simeq 0.003$--$0.01$);
(iv) strong azimuthal structure in footprint-corrected FRB distributions
($p \simeq 0$);
and (v) excess low-$\ell$ multipole power that persists after simple
footprint removal.

These revised results narrow and sharpen the set of physical anomalies that
remain after removing geometry-induced artefacts, and provide a clearer basis
for the unified cosmic-axis likelihood.

\section{Conclusions}
\label{sec:conclusions}

We have carried out a broad suite of tests on a sample of 600 fast
radio bursts, expressed in a common coordinate system tied to a
``unified'' axis that approximately aligns an FRB-based direction,
the CMB hemispherical asymmetry axis, and a sidereal modulation axis.
The analyses span axis alignment, radial and azimuthal structure,
multipole decompositions, redshift splits, survey-footprint
corrections, and cross-messenger comparisons.

Our main conclusions are as follows:
\begin{enumerate}
    \item \textbf{Axis clustering.} The FRB, CMB hemispherical, and
    sidereal axes form an unusually tight triple on the sky:
    the maximum pairwise separation is ${\sim}13.5^\circ$, and Monte
    Carlo tests show that only $\sim 10^{-4}$ of random triples are
    this tightly clustered. Under the assumption that these axes are
    drawn independently from isotropy, such clustering is highly
    unlikely.

    \item \textbf{Layered and anisotropic FRB sky.} In the unified-axis
    frame, the FRB sky exhibits a pronounced shell-like radial profile
    with a break near ${\sim}25^\circ$ and a strong excess in the
    $25^\circ$--$40^\circ$ band. The shell structure remains highly
    significant under a variety of nulls (including dipole-subtracted
    skies) and after masking Galactic, ecliptic, and supergalactic
    planes, indicating that it is not an artifact of a single low-order
    multipole or of local foreground geometry.

    \item \textbf{Non-axisymmetric, warped shell morphology.} Harmonic
    fits and a warped-shell model of the form
    $R(\phi) = R_0[1 + a\sin\phi + b\cos\phi + c\sin 2\phi + d\cos 2\phi]$
    show that the excess around the unified axis is strongly lopsided.
    A purely radial shell is rejected at very high significance in
    favour of an $m=1+m=2$ ``multipatch'' shell, with
    $\Delta\mathrm{AIC}\sim\mathcal{O}(10^2)$ and Monte Carlo
    $p\simeq 0$. The inferred shell radius varies by tens of degrees
    in azimuth, favouring an ``egg-shaped'' warped shell rather than
    a symmetric cone or ring.

    \item \textbf{Redshift stability.} Splitting the sample into
    low- and high-redshift halves yields consistent anisotropy:
    both subsets show strong shell structure and lopsided azimuthal
    dependence in the unified-axis frame. This disfavors a purely
    local (very low-redshift) origin and supports the interpretation
    of a large-scale, cosmologically extended anisotropy.

    \item \textbf{Limited cross-messenger support so far.} A neutrino
    axis reconstruction using the available high-energy neutrino events
    yields a dipole amplitude consistent with isotropy and an axis far
    from the unified FRB direction. At present, the neutrino data do
    not show a statistically significant alignment with the FRB axis,
    and therefore provide neither strong support nor strong tension;
    they are best regarded as a null cross-check at current statistics.

    \item \textbf{Unified likelihood.} Aggregating nominally
    independent $p$-values from axis clustering, radial break
    significance, width-layering tests, azimuthal structure, and
    low-$\ell$ multipole excess yields a combined
    $-\log_{10}p\approx 24.8$ under an isotropic sky hypothesis.
    While this combined likelihood must be interpreted with caution
    (because of possible correlations between tests and look-elsewhere
    effects), the overall pattern points consistently away from
    a strictly isotropic FRB sky and towards a structured, preferred
    direction.
\end{enumerate}

In summary, the present FRB sample favours a picture in which
fast radio bursts populate a warped, lopsided shell around a unified
axis on the sky, with structure extending across redshift and
remaining robust under multiple coordinate masks and null tests.
Whether this represents a new form of large-scale anisotropy in the
Universe, a non-trivial combination of survey selection functions,
or some as-yet-unidentified systematic remains an open and testable
question.

\section{Future prospects and observational tests}
\label{sec:future}

The results presented here suggest several concrete directions for
further investigation:

\begin{itemize}
    \item \textbf{Larger and better-characterized FRB samples.}
    The most direct test is repetition of the analysis on larger,
    more homogeneous FRB catalogs with well-documented beam patterns
    and selection functions. In particular, separate analyses by
    instrument, frequency band, and detection pipeline can help
    discriminate between survey-specific footprints and genuinely
    cosmological structure.

    \item \textbf{Redshift tomography.}
    As more FRBs obtain host identifications and redshift measurements,
    the anisotropy can be probed in multiple redshift slices, enabling
    a tomographic test of whether the warped-shell morphology evolves
    with distance. A purely local structure should weaken rapidly
    beyond the scale of nearby superclusters, whereas a cosmological
    anisotropy should persist or grow with redshift.

    \item \textbf{Cross-correlations with large-scale structure.}
    The unified axis and the lopsided shell can be compared against
    dipoles and low-$\ell$ modes in galaxy, quasar, and radio source
    surveys, as well as against bulk-flow or peculiar-velocity
    measurements. A significant alignment with known large-scale
    structure would favor an astrophysical explanation; a lack of
    correlation would push the interpretation towards more exotic
    possibilities or subtle systematics.

    \item \textbf{Multi-messenger extensions.}
    The neutrino analysis presented here is limited by small-number
    statistics. Future high-energy neutrino samples, as well as
    ultra-high-energy cosmic rays and other transient populations,
    can be folded into the axis-stacking framework to test whether
    any other messenger shares the FRB-preferred direction or shell
    geometry.

    \item \textbf{Refined modelling of selection effects.}
    A key systematic uncertainty remains the detailed survey footprint
    and detection efficiency as a function of direction, frequency,
    and instrumental configuration. Incorporating realistic beam
    models and end-to-end simulations into the analysis will be
    essential to separate intrinsic anisotropy from subtle
    direction-dependent sensitivity.

    \item \textbf{Physical modelling of warped shells.}
    On the theory side, the warped-shell morphology inferred here
    provides a concrete target for models in which FRB sources trace
    anisotropic large-scale structure, cosmic voids, or more exotic
    sectors. Forward-modelling of such scenarios in the unified-axis
    frame, including realistic survey masks, can test whether any
    physically motivated model can reproduce both the radial layering
    and the strong $m=1+m=2$ azimuthal structure.
\end{itemize}

As FRB samples grow and multi-wavelength, multi-messenger data improve,
the framework developed here can be iteratively refined. The essential
question is whether the warped, lopsided shell seen in the current data
persists as a stable feature, resolves into survey systematics, or
transitions into a more complex anisotropy field. In each case, the
answer will place non-trivial constraints on the isotropy of the
Universe as probed by fast radio bursts.


\end{document}
