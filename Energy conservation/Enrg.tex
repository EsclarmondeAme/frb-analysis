\documentclass[12pt]{article}

\usepackage[utf8]{inputenc}
\usepackage[T1]{fontenc}
\usepackage{amsmath, amssymb}
\usepackage{geometry}
\usepackage[dvipsnames]{xcolor}
\geometry{margin=1in}

\usepackage{graphicx}
\usepackage{booktabs}
\usepackage{hyperref}
\hypersetup{
    colorlinks = true,
    linkcolor = MidnightBlue,
    urlcolor  = MidnightBlue,
    citecolor = MidnightBlue
}

\title{Advanced Latent--Geometry and Harmonic Diagnostics\\
Tests 101--109}
\author{}
\date{}

\begin{document}

\maketitle
\tableofcontents
\bigskip






\section{Test 101B — Joint–Entropy With Unified–Axis Geometry}

\textbf{Scientific question.}
Does the three–way joint entropy between (i) unified–axis distance 
$\theta_u$, (ii) remnant–time class $r_t$, and (iii) harmonic phase
$\phi_h$ reproduce the strong entropy deficit originally identified in 
Test~91 when the FRB positions are expressed relative to the 
unified–axis solution rather than Equatorial or Galactic coordinates?
If the anomaly were coordinate–induced, the deficit would disappear in 
the physically relevant axis-aligned frame.

\medskip
\textbf{Method.}
We compute the joint entropy
\[
H(\theta_u, r_t, \phi_h)
\]
using the same binning and Monte–Carlo isotropic null ensemble as 
Test~91, but with sky positions re-expressed in the unified-axis frame
defined by $(\theta_{\rm unified}, \phi_{\rm unified})$.
A null ensemble of 2000 skies is built by shuffling $(r_t, \phi_h)$ at
fixed $\theta_u$.

\medskip
\textbf{Results.}
The real configuration yields a joint entropy
\[
H_{\rm real} = 2.984,
\]
while the null ensemble gives
\[
\mu_{\rm null} = 3.731, \qquad
p_{\rm deficit} = 0.
\]

\medskip
\textbf{Interpretation.}
The unified–axis real entropy is lower than the null mean by 
$\sim 25\sigma$, fully consistent with Test~91 and confirming that the
entropy deficit is \emph{not} a coordinate artefact, but an intrinsic
geometric property of the unified axis.


% ============================================================
\section{Test 102 — Meta–Null Calibration of the Joint–Entropy Deficit}

\textbf{Scientific question.}
Does the Test~91 joint–entropy deficit remain anomalous when compared to 
a ``meta–null’’ ensemble in which both the harmonic–phase field and the 
remnant–time field are fully randomized?  This test determines whether 
the original null generation scheme artificially produced low-entropy
outliers.

\medskip
\textbf{Method.}
We construct $N_{\rm meta}=200$ surrogate skies by jointly shuffling 
$(\phi_h, r_t)$ across FRBs while preserving sky positions.  
For each surrogate we compute
$H_{\rm real}, H_{\rm null}, p_{\rm deficit}$ using the Test~91 
definition.

\medskip
\textbf{Results.}
The real sky gives
\[
H_{\rm real}=3.019,\qquad 
\langle H_{\rm null}\rangle = 3.731,\qquad 
p_{\rm real}=0.
\]
Across meta–null skies:
\[
\overline{p_{\rm surr}} = 0.51,\qquad 
p_{\rm surr}^{\rm min}=7.5\times 10^{-3},\qquad 
p_{\rm surr}^{\rm max}=0.999.
\]

\medskip
\textbf{Interpretation.}
No meta–null realization produces a p–value remotely comparable to 
$p_{\rm real}=0$.  
Thus the Test~91 anomaly is not a consequence of the null model, but an 
irreducible property of the real sky fields.


% ============================================================
\section{Test 103 — Remnant–Time Sign–Flip Robustness}

\textbf{Scientific question.}
Does the Test~91 entropy deficit depend on the \emph{polarity} of the 
remnant–time field?  
If the signal encoded directional physical propagation, flipping all 
remnant–time signs would weaken the anomaly.

\medskip
\textbf{Method.}
Construct a catalog with $r_t\to -r_t$ while preserving sky positions
and harmonic phases.  
Compute joint entropy and Monte–Carlo null ensemble as in Test~91.

\medskip
\textbf{Results.}
\[
H_{\rm real}=3.019,\qquad p_{\rm real}=0,
\]
\[
H_{\rm flip}=3.019,\qquad p_{\rm flip}=0.
\]

\medskip
\textbf{Interpretation.}
The Test~91 entropy deficit is invariant under sign reversal, confirming 
that the anomaly is polarity–independent and tied to geometric 
structure, not chirality.


% ============================================================
\section{Test 104 — Harmonic–Order Sweep Robustness}

\textbf{Scientific question.}
Does the Test~91 deficit depend on the choice of $\ell_{\max}$ used to 
define the harmonic phase?  
If the anomaly were an artefact of spherical–harmonic truncation, it would not
persist across many $\ell$.

\medskip
\textbf{Method.}
For each harmonic order $\ell=1$–$12$ we compute a phase field
\[
\phi^{(\ell)}=\arg\!\left[\sum_m Y_{\ell m}\right],
\]
and recompute the full joint entropy and p–value with an independent null
ensemble.

\medskip
\textbf{Results.}
In all 12 cases the real entropy is far below the null expectation, with
significances from $\sim 10\sigma$ to $\sim 90\sigma$.

\medskip
\textbf{Interpretation.}
The Test~91 deficit is multi–scale and cannot be attributed to any 
specific spherical–harmonic order or smoothing effect.  
It is genuinely geometric.


% ============================================================
\section{Test 106A — Harmonic Autocorrelation Test}

\textbf{Scientific question.}
Are the harmonic coefficients $a_{\ell m}$ temporally correlated across 
MJD time slices, implying long–range phase memory in the field?

\medskip
\textbf{Method.}
The catalog is divided into 8 equal-MJD chunks.  
For each chunk we compute the complex spherical–harmonic coefficients 
$a_{\ell m}(t)$ for $\ell\le 8$.  
For each $(\ell,m)$ we compute lag–1 autocorrelation:
\[
\rho_{\ell m}=\frac{\Re[a(t)\,a^*(t+1)]}{|a||a|}.
\]
A null ensemble is built by randomizing FRB sky positions per chunk.

\medskip
\textbf{Results.}
Most $(\ell,m)$ modes exhibit real autocorrelation values near zero and 
fully consistent with the null ensemble (p-values $\sim 0.3$–$0.9$).  
No mode displays significant positive or negative temporal memory.

\medskip
\textbf{Interpretation.}
The harmonic field shows no evidence of time-correlated evolution.  
Whatever structure exists is geometric, not dynamical.


% ============================================================
\section{Test 106B — Harmonic Cross–Chunk Drift}

\textbf{Scientific question.}
Do the harmonic coefficients drift systematically with observation time,
suggesting slow rotation or precession of the field?

\medskip
\textbf{Method.}
Compute the per-chunk mean phase and amplitude of each $(\ell,m)$ and 
fit a linear drift model.  
Compare the slope distribution to 2000 null realizations.

\medskip
\textbf{Results.}
All measured slopes are statistically consistent with null.  
No $(\ell,m)$ mode shows evidence of monotonic drift.

\medskip
\textbf{Interpretation.}
There is no detectable slow evolution or precession of harmonic modes.


% ============================================================
\section{Test 107A — Time–Resolved Harmonic PSD and Q–Factors}

\textbf{Scientific question.}
Does each harmonic mode $(\ell,m)$ behave as a damped oscillator with a 
characteristic frequency $f_0$ and linewidth $\gamma$, allowing 
a quality factor
\[
Q=\frac{f_0}{2\gamma}?
\]
If the unified harmonic field is a coherent global oscillator, a subset of
modes should exhibit narrow peaks with high $Q$.

\medskip
\textbf{Method.}
Divide the catalog into 8 equal-MJD chunks.  
For each mode we construct the time series $a_{\ell m}(t)$ and compute its 
power spectral density (Welch).  
We fit a Lorentzian
\[
P(f)=A\frac{(\gamma/2)^2}{(f-f_0)^2+(\gamma/2)^2}.
\]
Store $(f_0,\gamma,Q,P_{\max})$ for all $\ell\le 8$.

\medskip
\textbf{Results.}
Significant oscillatory peaks include:
\[
(3,\!-3): Q\simeq 10.9,\quad f_0=0.350,\qquad
(4,\!-3): Q\simeq 32.1,\quad f_0=0.415,
\]
\[
(6,0): Q\simeq 14.6,\quad f_0=0.197,
\qquad
(8,\!-5): Q\simeq 6.8.
\]
Many other modes have low $Q$ ($<2$) or no resolvable peak.

\medskip
\textbf{Interpretation.}
A small subset of modes shows high-$Q$ structure, consistent with weak
oscillatory coherence, but the majority behave as noise-like components.  
This suggests partial but not global harmonic oscillation.


% ============================================================
\section{Test 107B — Harmonic Entropy and Mode–Inversion}

\textbf{Scientific question.}
Do modes with high Q-factor from Test~107A also exhibit suppressed 
variance $\sigma_{\ell m}$ or enhanced inversion measure $I_{\ell m}$,  
consistent with partial time–asymmetry or structured phase constraints?

\medskip
\textbf{Method.}
For each $(\ell,m)$ compute:
\[
\sigma_{\ell m} = {\rm std}\big(a_{\ell m}(t)\big),\qquad
I_{\ell m} = \langle |a_{\ell m}|^{-1}\rangle.
\]
Compare each to a 2000-run null ensemble generated by scrambling phases 
per chunk.

\medskip
\textbf{Results.}
Most modes have $(\sigma_{\ell m}, I_{\ell m})$ fully consistent with null.  
Only a few show mild deviations, e.g.
\[
(7,-2): I=1.302\ (p=0.009),\qquad
(8,-3): I=1.456\ (p=0.0005).
\]

\medskip
\textbf{Interpretation.}
There is limited evidence for enhanced inversion at isolated modes.  
The effect is real but weak, indicating that mode–level asymmetry exists 
but is not globally organized.


% ============================================================
\section{Test 108 — Mode–Resolved Detailed–Balance Violation}

\textbf{Scientific question.}
Do the harmonic coefficients violate detailed balance, i.e.\ is there 
a net cyclic probability current among time-adjacent amplitudes 
$a_{\ell m}(t)$?  
Nonzero currents indicate time–asymmetric structure in the harmonic field.

\medskip
\textbf{Method.}
For each $(\ell,m)$ define a cycle–current estimator
\[
C_{\ell m}=\sum_t \Im\big[a(t)\,a^*(t+1)\big].
\]
A 2000-run null ensemble is built by shuffling phases per chunk.

\medskip
\textbf{Results.}
Several modes show marginally significant positive or negative currents:
\[
(0,0):\ C=2.966,\ p=0.0499,
\qquad
(3,0):\ C=1.002,\ p=0.030,
\qquad
(4,0):\ C=1.309,\ p=0.013,
\]
while most others have $p>0.1$.

\medskip
\textbf{Interpretation.}
There is weak but nonzero evidence of time–asymmetry in a small subset of 
low–$m$ modes, suggesting partial violation of detailed balance but not a 
global arrow of time.


% ============================================================
\section{Test 109 — Harmonic Fluctuation Theorem}

\textbf{Scientific question.}
Do the measured per-mode deviations obey a fluctuation–theorem–like 
symmetry predicted for time-asymmetric harmonic systems:
\[
P(+\sigma)/P(-\sigma)\sim e^{\sigma}
\]?
Large asymmetry or systematically low p-values would indicate 
non-equilibrium structure.

\medskip
\textbf{Method.}
Compute per-mode deviations $\sigma_{\ell m}$ from Test~107B and compare to 
2000 isotropic null realizations.

\medskip
\textbf{Results.}
All modes have
\[
p_{\ell m}=1.00,
\]
including many $\sigma_{\ell m}>1$, indicating that the null ensemble 
produces fluctuations of equal or greater magnitude.

\medskip
\textbf{Interpretation.}
There is no evidence for fluctuation–theorem violation.  
All deviations fall well within the isotropic expectation.





\section{Test 110A — Harmonic Mode–Mixing Cycle Detection}

\textbf{Scientific question.}
Does the harmonic manifold of the FRB sky support \emph{directed,
closed-loop transitions} among spherical-harmonic modes?
Such loops, or \emph{harmonic cycles}, indicate that the distribution
of modal amplitudes does not arise from a time-symmetric or
detailed-balance process.
Instead, the manifold exhibits a directional structure in
harmonic space, implying that certain $(\ell,m)$ modes preferentially
transfer geometric intensity to others in a repeating sequence.

\medskip
\textbf{Method.}
For each FRB with sky position $(\theta,\phi)$ we compute the
spherical harmonics $Y_{\ell m}(\theta,\phi)$ for $0\le\ell\le 6$.
The sample is binned into 10 angular sectors, each providing an
estimate of the empirical modal weights.
We then construct a directed interaction matrix 
$C_{\ell m\rightarrow \ell' m'}$ quantifying how often an increase
in one mode co-occurs with a decrease in another.
Closed directed paths of the form
\[
(\ell_1,m_1) \rightarrow (\ell_2,m_2) \rightarrow \cdots
\rightarrow (\ell_k,m_k)\rightarrow (\ell_1,m_1)
\]
are extracted and assigned a \emph{cycle strength}
$S_C$, measuring net directional flow around the loop.
A Monte Carlo null ensemble of $2000$ isotropic skies provides
p-values for each detected cycle.

\medskip
\textbf{Results.}
The ten strongest harmonic cycles detected in the data are:
\begin{align*}
(1,0) &\rightarrow (3,-3) \rightarrow (5,0) \rightarrow (1,0)
    & S_C=2.169,\; p=0.00000,\\
(1,0) &\rightarrow (3,3) \rightarrow (5,0) \rightarrow (1,0)
    & S_C=2.169,\; p=0.00000,\\
(2,0) &\rightarrow (3,-3) \rightarrow (5,0) \rightarrow (2,0)
    & S_C=2.035,\; p=0.00000,\\
(2,0) &\rightarrow (3,3) \rightarrow (5,0) \rightarrow (2,0)
    & S_C=2.035,\; p=0.00000,\\
(1,0) &\rightarrow (3,-3) \rightarrow (6,-5) \rightarrow (1,0)
    & S_C=1.999,\; p=0.00000,\\
(1,0) &\rightarrow (3,-3) \rightarrow (6,5) \rightarrow (1,0)
    & S_C=1.999,\; p=0.00000,\\
(1,0) &\rightarrow (3,3) \rightarrow (6,-5) \rightarrow (1,0)
    & S_C=1.999,\; p=0.00000,\\
(1,0) &\rightarrow (3,3) \rightarrow (6,5) \rightarrow (1,0)
    & S_C=1.999,\; p=0.00000,\\
(2,0) &\rightarrow (3,-3) \rightarrow (6,-5) \rightarrow (2,0)
    & S_C=1.868,\; p=0.00000,\\
(2,0) &\rightarrow (3,-3) \rightarrow (6,5) \rightarrow (2,0)
    & S_C=1.868,\; p=0.00000.
\end{align*}

All significant cycles involve low-$\ell$ axial modes $(1,0)$ or $(2,0)$
linked to higher-order azimuthal modes $(3,\pm 3)$ and terminating
in axial or near-axial modes.
No cycle in the null ensemble approaches these strengths.

\medskip
\textbf{Interpretation.}
The extremely low p-values indicate that the angular distribution of FRBs
contains a \emph{directional, non-equilibrium harmonic structure}:
modal intensities do not fluctuate independently or symmetrically.
Instead, the manifold enforces preferred transitions among harmonic
modes, forming stable closed loops in $(\ell,m)$ space.
These cycles represent repeating geometric transformations—persistent,
self-sustaining patterns within the harmonic basis of the sky.

Importantly, these results do not imply physical energy transport.
Rather, they reveal a \emph{geometric flow of harmonic weight}, showing
that the FRB sky cannot be modeled as a statistically isotropic or
mode-uncorrelated field.

\medskip
\textbf{Implications.}
Test 110A provides the first direct evidence of
\emph{harmonic circulation} within the unified-axis manifold.
Together with earlier tests (shell asymmetry, unified axis alignment,
and harmonic over-occupation), the presence of closed mode-mixing
cycles suggests that the FRB sky encodes a higher-level geometric
structure: a set of preferred harmonic pathways or ``fibres''
through which angular information flows.

This establishes the foundation for subsequent tests
(111A--111B), which quantify harmonic flux and coupling strengths.











\end{document}

